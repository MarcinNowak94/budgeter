% Docummentation: 
% - https://www.latex-project.org/help/documentation/
% - https://docs.w3cub.com/latex/

\documentclass[a4paper,12pt]{report}

\usepackage{polski}         % Polish diacretic signs
\usepackage[utf8]{inputenc} % required for international characters
\usepackage{hyperref}       % urls and hyperlinks
\usepackage{xurl}           % break urls
\usepackage{microtype}      % improve justification
\usepackage{enumitem}       % compact lists
\usepackage{graphicx}       % Include graphics
\usepackage{wrapfig}        % wrap text aroung graphics
\usepackage{fancyhdr}       % Customize page layout
\usepackage{index}          % Create an index
\usepackage{setspace}       % Spacing
\makeindex

% ------------------------------ Custom Commands ------------------------------
% Usage: \command\{text}  
\newcommand{\customstyletitle}[1]{\Huge{\textbf{#1}}}
\newcommand{\customstylechapter}[1]{\large{\textit{#1}}}
\newcommand{\customstylesection}[1]{\textbf{\textit{#1}}}
\newcommand{\customstylesidenote}[1]{\Small{\textbf{#1}}}

\newcommand{\HRule}{\rule{\linewidth}{0.5mm}} % horizontal lines

% --------------------------- documment starts here ---------------------------

% environment
\begin{document}

% Define pagestyle
% [REQUIREMENT] 23. Przypisy dolne, stopki (nr stron), nagłówki...
\pagestyle{fancy}
\fancyhf{}          %clears default headers and footers
\renewcommand{\headrulewidth}{2pt}
\renewcommand{\footrulewidth}{1pt}
%\fancyhead[LE]{\leftmark}
%\fancyhead[RO]{\rightmark}
\fancyfoot[RO,LE]{\thepage}


% [REQUIREMENT] 1. Strona tytułowa
\begin{titlepage}
	%---Headings------------------------------------------	
	\begin{center}
    \begin{onehalfspace}
    \textsc{\LARGE{WYŻSZA SZKOŁA TECHNOLOGII INFORMATYCZNYCH W KATOWICACH}}\\
    \end{onehalfspace}
    \textsc{\large{WYDZIAŁ INFORMATYKI}}\\
	\textsc{\large{KIERUNEK: INFORMATYKA}}\\
    \end{center}
    
    %---Author--------------------------------------------
	\begin{flushleft}
    \textsc{Nowak Marcin}\\[0cm]
    \textsc{Nr Albumu 08255}\\[0cm]
    \textsc{Studia niestacjonarne}\\[0cm]
    \end{flushleft}
	
    %---Title---------------------------------------------
	\begin{center}
    \HRule\\[0.4cm]
	{\customstyletitle{Projekt aplikacji wspomagającej zarządzanie budżetem}}\\[0.4cm] 
    \HRule\\[1.5cm]
    \end{center}
	
    %---Author--------------------------------------------
	\begin{flushright}
        \textsc{Przedmiot: Projekt Systemu Informatycznego}\\[0cm]
        \textsc{pod kierunkiem}\\[0cm]
        \textsc{mgr. Jacek Żywczok}\\[0cm]
        \textsc{W roku akademickim 2022/23}\\[0cm]
    \end{flushright}
 
	%---Date & logo---------------------------------------
	\vfill                  % Position the date lower
	\begin{center}
    {Katowice 2022}\\	    % \today
	\includegraphics[width=0.2\textwidth]{WSTI-logo.jpg}\\[1cm]
	\end{center}
\end{titlepage}

% [REQUIREMENT] 2. Spis treści
\renewcommand*\contentsname{Spis treści}
\tableofcontents                    % prints automatical table of contents

% ---------------------------------- Content ----------------------------------

% [REQUIREMENT] 3. Wprowadzenie do tematyki projektu
\chapter{\customstylechapter{Wprowadzenie do tematyki projektu}}
{Finanse są dziedziną nauki ekonomicznej zajmującą się rozporzadzaniem pieniędzmi
 \cite{wiki_ekonomia}. Nauka ta w podobnym zakresie a różnej skali 
wykorzystywana jest tak przez rządy, przedsiębiorstwa jak i zwykłych obywateli - 
w efekcie jest to dziedzina o stosunkowo prostych podstawach jednak niesamowicie
skomplikowana w każdym zakresie w którym chętna osoba zadecyduje się ją zagłębić. 
Wiedza z zakresu finansów staje się szczególnie przydatna podczas gdy na rynku 
panuje trudna sytuacja ekonomiczna, w takich warunkach nierzadko decyduje ona o 
jakości oraz stanie życia poszczególnych osób fizycznych jak i całych 
przedsiębiorstw a nawet krajów.}
%
% [REQUIREMENT] 4. Zamierzony cel projektu
\chapter{\customstylechapter{Zamierzony cel projektu}}
{Celem projektu jest ułatwienie zarządzania finansami i budżetem poprzez 
uproszczenie analizy wpływów i wydatków dzięki wizualizacji trendów, 
automatycznej kategoryzacji wydatków i wpływów oraz predefiniowanym 
zestawieniom. Użytkownik po wprowadzeniu danych będzie w stanie w łatwy sposób 
zobrazować sytuację finansową osobistą lub przedsiębiorstwa co pozwoli 
bardziej świadomie podejmować dalsze decyzje finansowe, planować budżet, łatwo 
identyfikować obszary które wymagają usprawnień czy ogólną obserwację trendów.}
%
% [REQUIREMENT] 5. Wstępne założenia i uwarunkowania, w których 
% projekt będzie powstawał  -----------------------
\chapter{\customstylechapter{Wstępne założenia i uwarunkowania}}
\section{\customstylesection{Założenia}}
{Początkowo użytkownik będzie samodzielnie wprowadzał dane dane do aplikacji 
poprzez dedykowany interfejs. Aplikacja będzie dbać o jakość danych przyjmując 
jednak oznaczając i pomijając dane błędne, niepełne lub niepewne - będą one 
prezentowane w dedykowanej zakładce aplikacji gdzie użytkownik będzie mieć 
możliwość ich poprawy. Użytkownik będzie w stanie wybrać zestaw 
predefiniowanych typów i kategorii obiektów lub utworzyć i edytować własne. 
Aplikacja będzie udostępniać predefiniowane wizualizacje, wliczając możliwość 
wizualizacji określonego przez użytkownika obiektu.}
%
\section{\customstylesection{Uwarunkowania}}
{Celem projektu jest dostarczenie minimalnego spełniającego wymagania produktu, 
pozostałe funkcjonalności obecnie zostaną pominięte z powodu różnych czynników 
jak ograniczony czas wdrożenia, zakres umiejętności technicznych autora czy 
fakt że jest to projekt w głównej mierze edukacyjny.
Termin wdrożenia wyklucza bardziej zaawansowane funcjonalności, jako że jest to 
projekt edukacyjny znajomość technologii będzie budowana w trakcie co wpłynie 
między innymi na ograniczenia systemowe: wstępnie jest to aplikacja wyłacznie 
stacjonarna przeznaczona na systemy Windows 11. Kolejnym na ten moment 
pominiętym aspektem jest zabudowanie reguł przeprowadzających dogłębną analizę 
statystyczną danych które otwierają dalsze możliwości rozwoju oprogramowania. 
Aplikacja będzie także z zasady obsługiwać wyłącznie pojedynczego użytkownika, 
a zawarte w niej dane będa przechowywane wyłącznie lokalnie. Pominięte zostanie 
także automatyczne pobieranie danych z interfejsów innych aplikacji lub w 
formie ekstrakcji danych ze skanowanych dokumentów czy kodów EAN lub QR towarów.
 Aplikacja nie będzie takze udostępniać żadnego rodzaju interfejsu 
programistycznego (API). W momencie zakończenia projektu wszystkie dane 
użytkownika przechowywane będą w pojedynczym miejscu, w przyszłości może jednak 
zajść potrzeba rozdzielenia danych w aplikacji od konfiguracji użytkownika. 
Interfejs będzie zaprojektowany by wyglądać w z góry określony sposób, 
użytkownik nie będzie miał możliwości jego edycji.}
%
% [REQUIREMENT] 6. Założone ograniczenia (ramy czasowe, umiejętności) 
% i możliwość ewaluacji projektu
\chapter{\customstylechapter{Założone ograniczenia i możliwosci ewaluacji projektu}}
{W aplikacji utworzony zostanie panel administracyjny prezentujący użytkownikowi 
dane statystyczne prezentujące ilość, zakres i jakość danych a także sugerujące 
kolejny krok ich usprawnienia. Standard danych w aplikacji dopasowany zostanie 
do wiodącego globalnego standardu danych w obrembie tej samej tematyki. Typy 
obiektów będzie można grupować na kilku poziomach aby ułatwić użytkownikowi 
zarządzanie danymi i uprościć wizualizacje. Dla zaawansowanych użytkowników 
może okazać się przydatna możliwość definiowania i zapisywania własnych 
wizualizacji i raportów statystycznych - wymagać to będzie jednak implementacji 
dedykowanego modułu.}
% [REQUIREMENT] 7. Chronologiczny plan pracy (ujecie przyjetego modelu 
% projektowania i faz projektowania)
\chapter{\customstylechapter{Plan pracy}}
{Prace nad projektem prowadzone będą w formie listy zadań do zrealizowania 
którym przypisane zostaną priorytety metodą M.O.S.C.O.W. lub Matrycy 
Eisenhowera. Przewidywany plan pracy nad projektem prezentuje się następująco:
\setitemize{noitemsep,topsep=0pt,parsep=0pt,partopsep=0pt}
\begin{enumerate}
    \item Spis założeń w dokumentacji wstępnej
    \begin{itemize}
        \item Założenia wstępne
        \item Spis wymagań każdego typu
        \item Przegląd rynku pod kątem dostępnych rozwiązań
        \item Określenie metodologii pracy
        \item Dokumentacja modelowania
        \item Dokumentacja uruchomieniowa projektu
        \item Przeprowadzone testy
        \item Instrukcja obsługi dla użytkownika
        \item Retrospekcja
    \end{itemize}
    \item Modelowanie 
    \begin{itemize}
        \item Utworzenie słownika modelowanej domeny
        \item Określenie wymaganych kontenerów
        \item Określenie wymaganych encji i atrybutów
        \item Określenie wymaganych ograniczeń danych
        \item Modelowanie powiazań encji
    \end{itemize}
    \item Wybór technologii
    \begin{itemize}
        \item Wspierane systemy i wersje
        \item Wybór języka
        \item Biblioteki interfejsu użytkownika
        \item Sposób przechowywania danych
        \item Instalator, aktualizacja i utrzymanie 
    \end{itemize}
    \item Wstępne wdrożenie
    \begin{itemize}
        \item Utworzenie struktur bazy danych
        \item Wypełnienie danymi testowymi
        \item Podstawowe triggery i widoki
        \item Projekt interfejsu użytkownika
        \item Szkielet interfejsu użytkownika
        \item Połączenie interfejsu z bazą danych
        \item Podstawowa wizualizacja
        \item Iteracyjne uzupełnienie interfejsu i bazy o dodatkowe funkcje
        \item Usprawnienia
    \end{itemize}
    \item Testy rozwiązania
    \begin{itemize}
        \item Utworzenie danych testowych
        \item Określenie spodziewanych wyników
        \item Porównanie wyników oczekiwanych z otrzymanymi 
    \end{itemize}
    \item Iteracyjne usprawnienia projektu i uzupełnianie dokumentacji
    \item Retrospekcja
    \begin{itemize}
        \item Przydatność gotowej aplikacji
        \item Wady i zalety podejścia
        \item Sprawność rozwiązań
        \item Sprawność technologii
        \item Spis wniosków
    \end{itemize}
\end{enumerate}
}
% [REQUIREMENT] 22. Bibliografia - wykaz wszystkich źródeł
\begin{thebibliography} {books}
\bibitem{wiki_ekonomia} Wikipedia \raggedright\url{https://pl.wikipedia.org/wiki/Nauki_ekonomiczne}
\bibitem{gus_sytuacja_budzetowa} Główny Urząd Statystyczny \raggedright\url{https://stat.gov.pl/obszary-tematyczne/warunki-zycia/dochody-wydatki-i-warunki-zycia-ludnosci/sytuacja-gospodarstw-domowych-w-2021-r-w-swietle-badania-budzetow-gospodarstw-domowych,3,21.html}
\bibitem{o24_budzetowanie} Opcje24, Budzetowanie \raggedright\url{https://www.opcje24h.pl/budzetowanie-przewodnik-planowanie-budzetu/}
\end{thebibliography}

\end{document}



% ------------------------------ Docummentation -------------------------------

% ------ Project deadline 2023-01-29 ------------------------------------------
% ------ Part 1: Due 2022-10-21 

% [REQUIREMENT] 9. Wymagania funkcjonalne (szczegółowey wykaz wszystkich funkcji 
% oprogramowania)
% jakie funkcjonalności oprogramowania chcemy dostarczyć, można nie zdążyć z 
% dostarczeniem części, lub dopisać dodatkowe dodane w trakcie

% [REQUIREMENT] 10. Wymagania niefunkcjonalne 
% [REQUIREMENT] 10.1. Sprzętowe (w różnych wariantach, w tym dostęp do 
% koniecznych lub alternatywnych nośników danych i peryferiów)
% [REQUIREMENT] 10.2. Systemowe (systemy operacyjne, zainstalowane środowiska, 
% platformy, pakiety, biblioteki, sterowniki)
% [REQUIREMENT] 11.3. Organizacyjne (np. organizacja pracy z systemem, warunki 
% poprawnej pracy przy większej liczbie użytkownikó, stanowisk, obciążeniu sieci,
% konieczność zapewnienia realnego czasu dostępu itd.itp.)
% [REQUIREMENT] 11. Wymagania dotyczące danych (wykaz tabel, relacji, 
% typy i rozmiary pól z uzasadnieniem, inne rodzaje danych w tym logi, hasła)
% [REQUIREMENT] 12. Metody pracy, narzędzia i techniki
% [REQUIREMENT] 13. Opis głównych klas, metod, obiektów, struktur i algorytmów
% zastosowanych w projekcie (uwzględniając stsosowanie gotowych narzędzi
% obcego autorstwa, w tym open source)

% ------ Part 2: Due 2022-? 
% [REQUIREMENT] 14. POWYŻSZĄ CZĘŚĆ DOKUMENTACJI ODDAJEMY W TRAKCIE TWORZENIA

% [REQUIREMENT] 15. Przebieg uruchamiania projektu (być może na różnych 
% platformach, konfiguracjach...)
% [REQUIREMENT] 16. Przebieg testowania projektu (rodzaje i metody przeprowadzonych
% testów)
% [REQUIREMENT] 17. Wnioski z przebiegu testowania (wykryte defekty, wrażliwość
% na specyficzne dane, błędy ukryte i niewidoczne dla użytkownika, sytuacje
% niejednoznaczne itp.)
% [REQUIREMENT] 18. Konserwacja systemu
% [REQUIREMENT] 19. Podsumowanie i alternatywne sposoby stworzenia projektu
% (po zdobytym doświadczeniu, przy dostępie do innych narżedzi, przy innej wizji...)
% [REQUIREMENT] 20. Dokumentacja dla użytkownika (Podręczni kużytkownika)
% [REQUIREMENT] 20.1. Przeznaczenie i główne możliwości systemu
% [REQUIREMENT] 20.2. Podstawowe wymagania
% [REQUIREMENT] 20.3. Opis instalacji/uruchamiania
% [REQUIREMENT] 20.4. Kompletny opis działających funkcji (menu, opis interface...),
% formatów danych, obsługi błędów użytkowania, zakresów danych 
% [REQUIREMENT] 20.5. Podręcznik administratora/użytkownika systemu/gościa
% [REQUIREMENT] 20.6. Spostrzeżenia i zalecenia do użytkowania projektu
% [REQUIREMENT] 20.7. Wykryte błędy w działaniu
% [REQUIREMENT] 21. Spisy ilustracji (spis obiektów graficznych, diagramów, tabel...)


% DONE
% [REQUIREMENT] 1. Strona tytułowa
% [REQUIREMENT] 2. Spis treści
% [REQUIREMENT] 3. Wprowadzenie do tematyki projektu
% [REQUIREMENT] 4. Zamierzony cel projektu
% [REQUIREMENT] 5. Wstępne założenia i uwarunkowania, w których 
% projekt będzie powstawał  -----------------------
% [REQUIREMENT] 6. Założone ograniczenia (ramy czasowe, umiejętności) 
% i możliwość ewaluacji projektu
% [REQUIREMENT] 7. Chronologiczny plan pracy (ujecie przyjetego modelu 
% projektowania i faz projektowania)
% [REQUIREMENT] 8. POWYŻSZĄ CZĘŚĆ DOKUMENTACJI ODDAJEMY PRZED REALIZACJĄ PROJEKTU
% [REQUIREMENT] 23. Przypisy dolne, stopki (nr stron), nagłówki...


% --------------------------------- Purgatory ----------------------------------
% Orphaned snippets of texts placed here untill you decide what to do with them
% -----
% domowym i udostępnienie podstawowych informacji finansowych Niestety jedak 
% zakres wiedzy w zakresie finansów którą posiada przeciętny obywatel jest dość 
% niski, w połączeniu z obecnie panującą trudną sytuacją gospodarczą w wielu 
% miejscach na świecie część ludzi nie radzi sobie z dopinaniem budżetu 
% domowego. Jest to jednak dziedzina dość skomplikowana, a tym samym trudna dla 
% przeciętnego obywatela.