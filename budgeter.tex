\documentclass[a4paper,12pt]{book}

\usepackage{polski}         % Polish diacretic signs
\usepackage[utf8]{inputenc}
\usepackage{graphicx}       % Include graphics
\usepackage{wrapfig}        % wrap text aroung graphics
\usepackage{fancyhdr}       % Customize page layout
\usepackage{index}          % Create an index
\makeindex



% ------------------------------ Custom Commands ------------------------------
% Usage: \command\{text}  
\newcommand{\customstyletitle}[1]{\Large{\textbf{#1}}}
\newcommand{\customstylechapter}[1]{\Large{\textit{#1}}}
\newcommand{\customstylesection}[1]{\textbf{\textit{#1}}}

% --------------------------- documment starts here ---------------------------

% environment

\begin{document}
% [REQUIREMENT] 1. Strona tytułowa
\title{\customstyletitle{Aplikacja do budżetu}}
\author{Marcin Nowak}
\date{}

\maketitle                          % Prints title, author and date
\let\cleardoublepage\clearpage

% [REQUIREMENT] 2. Spis treści
\renewcommand*\contentsname{Spis treści}
\tableofcontents                    % prints automatical table of contents

% Define pagestyle
% [REQUIREMENT] 23. Przypisy dolne, stopki (nr stron), nagłówki...
\pagestyle{fancy}
\fancyhf{}          %clears default headers and footers
\renewcommand{\headrulewidth}{2pt}
\renewcommand{\footrulewidth}{1pt}
\fancyhead[L]{\leftmark}
\fancyhead[RO]{\rightmark}
\fancyfoot[L,RO]{\thepage}

% ---------------------------------- Content ----------------------------------

% [REQUIREMENT] 3. Wprowadzenie do tematyki projektu
\chapter{\customstylechapter{Wprowadzenie do tematyki projektu}}

% [REQUIREMENT] 4. Zamierzony cel projektu
\chapter{\customstylechapter{Zamierzony cel projektu}}


% [REQUIREMENT] 5. Wstępne założenia i uwarunkowania, w których 
% projekt będzie powstawał  -----------------------
\chapter{\customstylechapter{Wstępne założenia i uwarunkowania}}

\section{\customstylesection{Założenia}}

\section{\customstylesection{Uwarunkowania}}


% [REQUIREMENT] 6. Założone ograniczenia (ramy czasowe, umiejętności) 
% i możliwość ewaluacji projektu
\chapter{\customstylechapter{Założone ograniczenia i możliwosci ewaluacji projektu}}

% [REQUIREMENT] 7. Chronologiczny plan pracy (ujecie przyjetego modelu 
% projektowania i faz projektowania)
\chapter{\customstylechapter{Plan pracy}}

\end{document}

% ------ Project deadline 2023-01-29 -----------------------------------------
% ------ Due 2022-0 

% [REQUIREMENT] 8. POWYŻSZĄ CZĘŚĆ DOKUMENTACJI ODDAJEMY PRZED REALIZACJĄ PROJEKTU

% [REQUIREMENT] 9. Wymagania funkcjonalne (szczegółowey wykaz wszystkich funkcji 
% oprogramowania)

% [REQUIREMENT] 10. Wymagania niefunkcjonalne 
% [REQUIREMENT] 10.1. Sprzętowe (w różnych wariantach, w tym dostęp do 
% koniecznych lub alternatywnych nośnikó danych i peryferiów)
% [REQUIREMENT] 10.2. Systemowe (systemy operacyjne, zainstalowane środowiska, 
% platformy, pakiety, biblioteki, sterowniki)
% [REQUIREMENT] 11.3. Organizacyjne (np. organizacja pracy z systemem, warunki 
% poprawnej pracy przy większej liczbie użytkownikó, stanowisk, obciążeniu sieci,
% konieczność zapewnienia realnego czasu dostępu itd.itp.)
% [REQUIREMENT] 11. Wymagania dotyczące danych (wykaz tabel, relacji, 
% typy i rozmiary pól z uzasadnieniem, inne rodzaje danych w tym logi, hasła)
% [REQUIREMENT] 12. Metody pracy, narzędzia i techniki
% [REQUIREMENT] 13. Opis głównych klas, metod, obiektów, struktur i algorytmów
% zastosowanych w projekcie (uwzględniając stsosowanie gotowych narzędzi
% obcego autorstwa, w tym open source)
% [REQUIREMENT] 14. POWYŻSZĄ CZĘŚĆ DOKUMENTACJI ODDAJEMY W TRAKCIE TWORZENIA
% [REQUIREMENT] 15. Przebieg uruchamiania projektu (być może na różnych 
% platformach, konfiguracjach...)
% [REQUIREMENT] 16. Przebieg testowania projektu (rodzaje i metody przeprowadzonych
% testów)
% [REQUIREMENT] 17. Wnioski z przebiegu testowania (wykryte defekty, wrażliwość
% na specyficzne dane, błędy ukryte i niewidoczne dla użytkownika, sytuacje
% niejednoznaczne itp.)
% [REQUIREMENT] 18. Konserwacja systemu
% [REQUIREMENT] 19. Podsumowanie i alternatywne sposoby stworzenia projektu
% (po zdobytym doświadczeniu, przy dostępie do innych narżedzi, przy innej wizji...)
% [REQUIREMENT] 20. Dokumentacja dla użytkownika (Podręczni kużytkownika)
% [REQUIREMENT] 20.1. Przeznaczenie i główne możliwości systemu
% [REQUIREMENT] 20.2. Podstawowe wymagania
% [REQUIREMENT] 20.3. Opis instalacji/uruchamiania
% [REQUIREMENT] 20.4. Kompletny opis działających funkcji (menu, opis interface...),
% formatów danych, obsługi błędów użytkowania, zakresów danych 
% [REQUIREMENT] 20.5. Podręcznik administratora/użytkownika systemu/gościa
% [REQUIREMENT] 20.6. Spostrzeżenia i zalecenia do użytkowania projektu
% [REQUIREMENT] 20.7. Wykryte błędy w działaniu
% [REQUIREMENT] 21. Spisy ilustracji (spis obiektów graficznych, diagramów, tabel...)
% [REQUIREMENT] 22. Bibliografia - wykaz wszystkich źródeł




% DONE
% [REQUIREMENT] 1. Strona tytułowa
% [REQUIREMENT] 2. Spis treści
% [REQUIREMENT] 3. Wprowadzenie do tematyki projektu
% [REQUIREMENT] 4. Zamierzony cel projektu
% [REQUIREMENT] 5. Wstępne założenia i uwarunkowania, w których 
% projekt będzie powstawał  -----------------------
% [REQUIREMENT] 6. Założone ograniczenia (ramy czasowe, umiejętności) 
% i możliwość ewaluacji projektu
% [REQUIREMENT] 7. Chronologiczny plan pracy (ujecie przyjetego modelu 
% projektowania i faz projektowania)

% [REQUIREMENT] 23. Przypisy dolne, stopki (nr stron), nagłówki...