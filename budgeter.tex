% Docummentation: 
% - https://www.latex-project.org/help/documentation/
% - https://docs.w3cub.com/latex/
% - https://tex.stackexchange.com/questions/455993/formatting-sql-code
%   https://www.overleaf.com/learn/latex/Code_listing#Code_styles_and_colours

%--------------------------------------------------------------------------------
%-----------------------------------   TODO   ----------------------------------
%--------------------------------------------------------------------------------
%Add in dcummentation 
% - class description - basic PySimpleGUI class
% - applicatin can become crossplatform in the future (check how installers work)
%
%Needed functions:
% - Save and read (persistent) configuration - https://www.pysimplegui.org/en/latest/cookbook/#recipe-the-demo-browser Recipe - Save and Load Program Settings ( https://www.pysimplegui.org/en/latest/#user-settings-api )
% - Data quarantine for bad inputs - store data that did not go through 
% validation and let user correct it [preferably also provide reason why it was quarantined] 
% - Data export to csv
%
%Optional functions:
% - Data statistics (records in each table, firstdate-lastdate, data quality)
% - Change GUI theme: add selectable list with themes, store it in config #DONE
% - synopsis for functions
% - built in manual
% - error handling
% - Localization - Language versions
% - logs (easy-ish): logfile.append(date, event)
% - fix in place table editting
% - allow user to remove data
% - (loose idea) VERY optional: create database programatically: 
% 	- dump database creation script into sql file
%	- create dummy Finances.sqlite wherever user wants
%	- use script as database initialisation
% - automatically deduce the encoding GetDataFromCSV
% - add tresholds to graph (automatic based on average, user specified)
% - add predicted spending next month (requires: to ProductSummary and 
%		ProductType Sumary add columns mean[median?] cost (Amount/Bought Times),
%	  AverageDailyCost (Amount/(lastbought-firstbought))
% - change used database technology to secure data, centralize and allow simultaneous users

\documentclass[a4paper,10pt, twoside]{report}

\usepackage{polski}         % Polish diacretic signs
\usepackage[utf8]{inputenc} % required for international characters
\usepackage{hyperref}       % urls and hyperlinks
\usepackage{xurl}           % break urls
\usepackage{microtype}      % improve justification
\usepackage{enumitem}       % compact lists
\usepackage{graphicx}       % Include graphics
\usepackage{wrapfig}        % wrap text aroung graphics
\usepackage{fancyhdr}       % Customize page layout
\usepackage{index}          % Create an index
\usepackage{setspace}       % Spacing
\usepackage{float}          % Forcing figure placement
\usepackage{tabularray}     % Tables with wrapping
\usepackage{xcolor,listings}% Code listings
\usepackage{textcomp}       % Code listings
\usepackage{color}          % Code listings
\usepackage{nameref}        % Reference chapter, section, etc. by name
\usepackage{afterpage}


\makeindex
\graphicspath{ {./} }       %Was ./figures/ changed so i can peek in VS Code


% Code listings
\definecolor{codegreen}{rgb}{0,0.6,0}
\definecolor{codegray}{rgb}{0.5,0.5,0.5}
\definecolor{codepurple}{HTML}{C42043}
\definecolor{backcolour}{HTML}{F2F2F2}
\definecolor{bookColor}{cmyk}{0,0,0,0.90}  
\color{bookColor}

\lstset{upquote=true}

\lstdefinestyle{mystyle}{
    backgroundcolor=\color{backcolour},   
    commentstyle=\color{codegreen},
    keywordstyle=\color{codepurple},
    numberstyle=\numberstyle,
    stringstyle=\color{codepurple},
    basicstyle=\footnotesize\ttfamily,
    breakatwhitespace=false,
    breaklines=true,
    captionpos=b,
    keepspaces=true,
    numbers=left,
    numbersep=10pt,
    showspaces=false,
    showstringspaces=false,
    showtabs=false,
}
\lstset{style=mystyle}

% Code listings
\newcommand\numberstyle[1]{%
    \footnotesize
    \color{codegray}%
    \ttfamily
    \ifnum#1<10 0\fi#1 |%
}

% ------------------------------ Custom Commands ------------------------------
% Usage: \command\{text}  
\newcommand{\customstyletitle}[1]{\Huge{\textbf{#1}}}
\newcommand{\customstylechapter}[1]{\large{\textit{#1}}}
\newcommand{\customstylesection}[1]{\textbf{\textit{#1}}}
\newcommand{\customstylesidenote}[1]{\Small{\textbf{#1}}}
\newcommand{\customstyletable}[1]{\footnotesize{\textbf{#1}}}
\newcommand{\customstyletablecentered}[1]{\footnotesize\centering{\textbf{#1}}}
\newcommand{\customstyleindivisible}[1]{
    \begin{minipage}{\textwidth}
        {#1}
    \end{minipage}
}

\lstnewenvironment{SQLlisting}[2][]%
  {\noindent\minipage{\linewidth}\medskip
   {#2}
   \smallskip
   \lstset{basicstyle=\ttfamily\footnotesize,
            frame=single,
            language=SQL,
            deletekeywords={IDENTITY},
            %deletekeywords={[2]INT},
            morekeywords={clustered},
            framesep=8pt,
            xleftmargin=40pt,
            framexleftmargin=40pt,
            frame=tb,
            framerule=0pt,
            caption=#1}}
  {\endminipage}

  % spacer
\newcommand{\HRule}{\rule{\linewidth}{0.5mm}} % horizontal lines



% --------------------------- documment starts here ---------------------------

% environment
\begin{document}

% Define pagestyle
% [REQUIREMENT] 23. Przypisy dolne, stopki (nr stron), nagłówki...
\pagestyle{fancy}
\fancyhf{}          %clears default headers and footers
\renewcommand{\headrulewidth}{2pt}
\renewcommand{\footrulewidth}{1pt}

\fancypagestyle{mychapterpage}{%
    %\fancyhead[LE]{\leftmark}
    %\fancyhead[RO]{\rightmark}
    \fancyfoot[RO,LE]{\thepage}
    \renewcommand{\headrulewidth}{2pt}
    \renewcommand{\footrulewidth}{1pt}
}


% https://texblog.org/2013/09/16/multiple-page-styles-with-fancyhdr/
%Redefine chapter by adding fancy as the chapter title page page-style
\makeatletter
    \let\stdchapter\chapter
    \renewcommand*\chapter{%
    \@ifstar{\starchapter}{\@dblarg\nostarchapter}}
    \newcommand*\starchapter[1]{%
        \stdchapter*{#1}
        \thispagestyle{mychapterpage}
        \fancyfoot[RO,LE]{\thepage}
        \markboth{\MakeUppercase{#1}}{}
    }
    \def\nostarchapter[#1]#2{%
        \stdchapter[{#1}]{#2}
        \thispagestyle{mychapterpage}
        \fancyfoot[RO,LE]{\thepage}
    }
\makeatother


% [REQUIREMENT] 1. Strona tytułowa
\begin{titlepage}
	%---Headings------------------------------------------	
	\begin{center}
    \begin{onehalfspace}
    \textsc{\LARGE{WYŻSZA SZKOŁA TECHNOLOGII INFORMATYCZNYCH W KATOWICACH}}\\
    \end{onehalfspace}
    \textsc{\large{WYDZIAŁ INFORMATYKI}}\\
	\textsc{\large{KIERUNEK: INFORMATYKA}}\\
    \end{center}
    
    %---Author--------------------------------------------
	\begin{flushleft}
    \textsc{Nowak Marcin}\\[0cm]
    \textsc{Nr Albumu 08255}\\[0cm]
    \textsc{Studia niestacjonarne}\\[0cm]
    \end{flushleft}
	
    %---Title---------------------------------------------
	\begin{center}
    \HRule\\[0.4cm]
	{\customstyletitle{Projekt i implementacja aplikacji wspomagającej zarządzanie budżetem domowym}}\\[0.4cm] 
    \HRule\\[1.5cm]
    \end{center}
	
    %---Description----------------------------------------
	\begin{flushright}
        \textsc{Przedmiot: Projekt Systemu Informatycznego}\\[0cm]
        \textsc{pod kierunkiem}\\[0cm]
        \textsc{mgr. Jacek Żywczok}\\[0cm]
        \textsc{W roku akademickim 2022/23}\\[0cm]
    \end{flushright}
 
	%---Date & logo---------------------------------------
	\vfill                  % Position the date lower
	\begin{center}
    {Katowice 2022}\\	    % \today
	\includegraphics[width=0.2\textwidth]{figures/WSTI-logo.jpg}\\[1cm]
	\end{center}
\end{titlepage}

%TODO: fix, this moves formatting 
\null\newpage %TODO: fix, this moves formatting \addtocounter{page}{-1}

% [REQUIREMENT] 2. Spis treści
\renewcommand*\contentsname{Spis treści}
\tableofcontents                    % prints automatical table of contents

% ---------------------------------- Content ----------------------------------

% [REQUIREMENT] 3. Wprowadzenie do tematyki projektu
\chapter{\customstylechapter{Wprowadzenie do tematyki projektu}}
{Finanse są dziedziną nauki ekonomicznej któa zajmuje się rozporządzaniem 
pieniędzmi \cite{wiki_ekonomia}. Nauka ta w podobnym zakresie a różnej skali 
dotyczy państw, przedsiębiorstw jak i zwykłych obywateli - w efekcie jest to 
dziedzina o stosunkowo prostych podstawach jednak niesamowicie skomplikowana w 
każdym zakresie w którym można ją zagłębić. Wiedza z zakresu finansów staje się 
szczególnie przydatna gdy na rynku panuje trudna sytuacja ekonomiczna, w takich 
warunkach nierzadko decyduje ona o jakości oraz stanie życia poszczególnych 
osób fizycznych, rentowności przedsiębiorstw czy stabilności państw. W przypadku
 państw i firma przeważnie budżetem zarządzają dedykowane osoby lub też całe 
zespoły, posiadające ekspercką wiedzę w tej dziedzinie. Jednak osoby 
zarządzające budżetem domowym najczęściej dysponują wyłacznie nabytym 
doświadczeniem, rzadko jeśi wogóle wspomagając się jakimikolwiek narzędziami 
które ułatwiałyby to zadanie.}
%
% [REQUIREMENT] 4. Zamierzony cel projektu
\chapter{\customstylechapter{Zamierzony cel projektu}}
{Celem projektu jest utworzenie modułu analitycznego aplikacji któa ułatwi 
zarządzanie budżetem domowym dostarczając użytkownikowi narzędzia do analizy 
wpływów i wydatków, wizualizacji trendów oraz automatycznie kategoryzujące 
wpływy i wydatki.} 

{Docelowymi odbiorcami aplikacji są użytkownicy domowi, możliwe że w miarę 
rozwoju w późniejszych fazach projektu także średnie lub małe przedsiębiorstwa. 
Użytkownik po wprowadzeniu danych uzyska dostęp do możliwie czytelnego obrazu 
sytuacji finansowej co uwidoczni trendy wydatków i przychodów, pozwoli bardziej
 świadomie podejmować dalsze decyzje finansowe, planować budżet a także łatwo
 identyfikować obszary które wymagają usprawnień. W efekcie uwidocznione przez
 aplikację informacje i wyciągnięte z nich wnioski ułatwią użytkownikowi poprawę
 sytuacji finansowej swojego domostwa poprzez bardziej efektywne zarządzanie
 budżetem.}
%
% [REQUIREMENT] 5. Wstępne założenia i uwarunkowania, w których 
% projekt będzie powstawał  -----------------------
\chapter{\customstylechapter{Wstępne założenia i uwarunkowania}}
\section{\customstylesection{Założenia}}
{Z uwagi na specyfikę tematyki potencjalnymi odbiorcami aplikacji będą osoby 
w niewielkim lub średnim stopniu zaznajomione z obsługą komputera. Największy
wpływ na projekt i rozwój graficznego interfejsu użytkownika oraz kierunek
 rozwoju aplikacji będzie miała łatwość obsługi. W efekcie aplikacja powinna
 być możliwie prosta, mieć przejrzysty, minimalistyczny interfejs dzięki 
czemu użytkownik nie zostanie przytłoczony mnogością dostępnych funkcji.}

{Początkowo użytkownik będzie wprowadzał dane do aplikacji samodzielnie poprzez 
dedykowany interfejs. Aplikacja zadba o jakość danych przyjmując jednak 
oznaczając i pomijając dane błędne, niepełne lub niepewne które zaprezentuje w 
dedykowanej zakładce gdzie użytkownik będzie mieć możliwość ich poprawy. 
Użytkownik będzie w stanie wybrać zestaw predefiniowanych typów i kategorii 
wydatkó lub utworzyć i edytować własne. Aplikacja udostępni predefiniowane 
wizualizacje, wliczając możliwość wizualizacji określonego przez użytkownika 
produktu lub całej kategorii produktów. We wstępnej wersji aplikacji interfejs
będzie statyczny, podobnie jak konfiguracja, która przechowywana będzie w pliku.}

\section{\customstylesection{Uwarunkowania}}
{Jest to projekt edukacyjny którego celem jest dostarczenie minimalnego
 opłacalnego produktu \cite{MVP}, swego rodzaju prototypu, na tym etapie funkcje
 dodatkowe zostaną pominięte ze względu na ograniczony czas produkcji, zakres
 umiejętności technicznych autora. Bliski termin oddania wyklucza bardziej
 zaawansowane funkcje, a znajomość technologii będzie budowana w trakcie jego
 rozwoju co wpłynie między innymi na ograniczenia systemowe. Aplikacja będzie
 także z zasady obsługiwać wyłącznie pojedynczego użytkownika, a dane przez
 niego wprowadzone będa przechowywane wyłącznie lokalnie. Pominięte zostanie
 także automatyczne pobieranie danych z interfejsów innych aplikacji lub w
 formie ekstrakcji danych ze skanowanych dokumentów czy kodów EAN lub QR
 produktów. Aplikacja nie będzie także udostępniać żadnego rodzaju interfejsu 
programistycznego (API). Założono że na tym etapie interfejs aplikacji będzie statyczny bez
 możliwości zmiany przez użytkownika.}

%Removed due to overspill:
% W momencie zakończenia projektu wszystkie dane
% użytkownika przechowywane będą w pojedynczym miejscu, w przyszłości może 
% jednak zajść potrzeba rozdzielenia danych w aplikacji od konfiguracji 
% użytkownika.

% [REQUIREMENT] 6. Założone ograniczenia (ramy czasowe, umiejętności) 
% i możliwość ewaluacji projektu
%[TODO] Shorten title to FIX overspill in table of contents 
\chapter{\customstylechapter{Założone ograniczenia i możliwości ewaluacji projektu}}
{W miarę możliwości standard danych w aplikacji dopasowany zostanie do wiodącego
 globalnego standardu danych w obrębie tej samej tematyki. Typy obiektów będzie
 można grupować na kilku poziomach aby ułatwić użytkownikowi zarządzanie danymi
 i uprościć wizualizacje. Dla zaawansowanych użytkowników może okazać się 
przydatna możliwość definiowania i zapisywania własnych wizualizacji i raportów 
statystycznych - wymagać to będzie jednak implementacji dedykowanego modułu. 
Kolejnym obecnie pominiętym aspektem jest zabudowanie reguł przeprowadzających 
dogłębną analizę statystyczną danych które otwierają dalsze możliwości rozwoju 
oprogramowania. Aplikacja dostarczana będzie użytkownikom w formie spakowanej w
archiwum skompilowanej pełnej wersji, jeśli zajdzie taka potrzeba i pojawi się 
możliwość stworzony zostanie także instalator. Możliwe że w aplikacji utworzony 
zostanie panel administracyjny prezentujący użytkownikowi dane statystyczne 
obrazujące ilość, zakres i jakość danych a także sugerujące kolejny krok ich 
usprawnienia.}

{Funkcjonalności importu i eksportu danych ze standardowych formatów będzie 
przydatna dla użytkownika podczas korzystania z projektu, wymaga określenia 
odpowiedniego formatu i standardu plików co może zająć sporo czasu dlatego 
zostały uznane za dodatkowe przez co możliwe że nie zostaną wdrożone w 
początkowej fazie projektu.}

{Obrane podejście pozwoli na rozwój aplikacji w różnych kiernkach, zależnie od 
opinii użytkowników. Aplikacja może zmienić model z aplikacji lokalnej która 
wspiera pojedynczego użytkownika na centralny obsługujących wielu użytkowników 
zdalnie. Ujednolicony interfejs umożliwia także powstanie potencjalnej wersji 
mobilnej co otworzy nowe możliwości wprowadzania danych oraz interakcji 
użytkownika z aplikacją.}
% [REQUIREMENT] 7. Chronologiczny plan pracy (ujecie przyjetego modelu 
% projektowania i faz projektowania)
\chapter{\customstylechapter{Plan pracy}}
{W toku prac stworzona zostanie lista zadań do zrealizowania, do określenia ich 
priorytetu posłuży metoda MoSCoW \cite{MOSCOW} lub Matryca Eisenhowera 
\cite{MatrycaEisenhowera}. Przewidywany plan pracy nad projektem prezentuje się następująco:
\setitemize{noitemsep,topsep=0pt,parsep=0pt,partopsep=0pt}
\begin{enumerate}
    \item Spis założeń w dokumentacji wstępnej
    \begin{itemize}
        \item Założenia wstępne
        \item Spis wymagań każdego typu
        \item Przegląd rynku pod kątem dostępnych rozwiązań
        \item Określenie metodologii pracy
        \item Dokumentacja modelowania
        \item Dokumentacja uruchomieniowa projektu
        \item Przeprowadzone testy
        \item Instrukcja obsługi dla użytkownika
        \item Retrospekcja
    \end{itemize}
    \item Modelowanie 
    \begin{itemize}
        \item Utworzenie słownika modelowanej domeny
        \item Określenie wymaganych kontenerów
        \item Określenie wymaganych encji i atrybutów
        \item Określenie wymaganych ograniczeń danych
        \item Modelowanie powiazań encji
    \end{itemize}
    \item Wybór technologii
    \begin{itemize}
        \item Wspierane systemy i wersje
        \item Wybór języka
        \item Biblioteki interfejsu użytkownika
        \item Sposób przechowywania danych
        \item Instalator, aktualizacja i utrzymanie 
    \end{itemize}
    \item Wstępne wdrożenie
    \begin{itemize}
        \item Utworzenie bazy danych
        \item Utworzenie podstawowych struktur bazy danych - tabele
        \item Wypełnienie danymi testowymi
        \item Utworzenie złożonych struktur bazy danych - widoki
        \item Projekt interfejsu użytkownika
        \item Szkielet interfejsu użytkownika
        \item Połączenie interfejsu z bazą danych
        \item Projekt podstawowych wizualizacji
        \item Iteracyjna uzupełnienie interfejsu i bazy o dodatkowe funkcje
        \item Usprawnienia i refaktoryzacja
    \end{itemize}
    \item Testy rozwiązania
    \begin{itemize}
        \item Utworzenie danych testowych
        \item Określenie spodziewanych wyników
        \item Porównanie wyników oczekiwanych z otrzymanymi
    \end{itemize}
    \item Iteracyjne usprawnienia projektu i uzupełnianie dokumentacji
    \item Retrospekcja
    \begin{itemize}
        \item Przydatność gotowej aplikacji
        \item Wady i zalety podejścia
        \item Sprawność rozwiązań
        \item Sprawność technologii
        \item Spis wniosków
    \end{itemize}
\end{enumerate}
}

% [REQUIREMENT] 9. Wymagania funkcjonalne (szczegółowey wykaz wszystkich funkcji 
% oprogramowania) jakie funkcjonalności oprogramowania chcemy dostarczyć, można 
% nie zdążyć z dostarczeniem części, lub dopisać dodatkowe dodane w trakcie
\chapter{\customstylechapter{Wymagania funkcjonalne}} 
\label{Wymagania funkcjonalne}
{Zestawienie funkcji które powinien spełniać program, wraz z informacją któe 
z nich zostały spełnione. Nagłówki z powodu objętości zostały skrócone, legenda:}

{PRIO - Priorytet w jednej z kategorii MOSCOW \cite{MOSCOW}}

{IMPL - Oznaczenie czy wdrożono funkcjonalność}

%[TODO]: try tabularray as suggested
% Wrapping as per: https://stackoverflow.com/questions/790932/how-to-wrap-text-in-latex-tables
\begin{table}[h]
    \footnotesize
    \begin{tabular}{|p{0.2\linewidth}|p{0.07\linewidth}|p{0.07\linewidth}|p{0.52\linewidth}|}  % | draws verical line
    % \usepackage{booktabs} provides different line thicknesses
    % \toprule, \midrule, \bottomrule
    \hline                  % Draw horizontal line
        
    % & Defines the breaks in the table 
    \customstyletable{Funkcjonalność} & \customstyletablecentered{PRIO} & \customstyletablecentered{IMPL}& \customstyletable{Opis} \\
    \hline
    {Plik konfiguracji} & {M} & {TAK} & {Osobny plik konfiguracyjny}\\
    \hline
    {Dodawanie danych} & {M} & {-} & {Dodawanie danych}\\
    \hline
    {Podsumowanie wydatków} & {M} & {TAK} & {Okresowe podsumowanie wydatków}\\
    \hline
    {Podsumowanie przychodów} & {M} & {TAK} & {Okresowe podsumowanie przychodów}\\
    \hline
    {Statystyki typów} & {C} & {TAK} & {Statystyki wydatków na dany typ produktu}\\
    \hline
    {Statystyki produktów} & {C} & {TAK} & {Statystyki wydatków na dany produkt}\\
    \hline
    {Bilans okresowy} & {M} & {TAK} & {Okresowy bilnas zysków i strat}\\
    \hline
    {Definiowanie produktów} & {M} & {TAK} & {Definiowanie produktów}\\
    \hline
    {Definiowanie przychodów} & {M} & {TAK} & {Definiowanie przychodów}\\
    \hline
    {Definiowanie typów produktów} & {M} & {TAK} & {Definiowanie typów produktów}\\
    \hline
    {Definiowanie typów przychodów} & {C} & {NIE} & {Definiowanie typów przychodów}\\
    \hline
    {Panel konfiguracyjny} & {S} & {TAK} & {Osobny panel konfiguracyjny}\\
    \hline
    {Rejestr zdarzeń} & {S} & {NIE} & {Logi z działania aplikacji}\\
    \hline
    {Dostęp zdalny} & {C} & {NIE} & {Dostęp do zdalnych baz danych}\\
    \hline
    {Import danych} & {C} & {TAK} & {Import danych w standardowym formacie}\\
    \hline
    {Walidacja danych} & {C} & {-} & {Potwierdzenie jakości danych}\\
    \hline
    {Eksport danych} & {C} & {-} & {Eksport danych do standardowego formatu}\\
    \hline
    {Instalator} & {C} & {NIE} & {Prosty instalator aplikacji}\\
    \hline
    {Trendy} & {W} & {NIE} & {Predykcja trendów wydatkó i wpływów}\\
    \hline
    {Porady} & {W} & {NIE} & {Porady dla użytkownika dotyczące usprawnień budżetu}\\
    \hline
    {Wiele użytkowników} & {W} & {NIE} & {Wsparcie dla wielu użytkowników jednocześnie}\\
    \hline
    {Personalizacja interfejsu} & {W} & {TAK} & {Personalizacja interfejsu użytkownika}\\
    \hline
    {Aktualizacje} & {W} & {NIE} & {Automatyczne sprawdzanie wersji i aktualizacja}\\
    \hline
    \end{tabular}
    \caption{Wymagania funkcjonalne}
\end{table}

% [REQUIREMENT] 10. Wymagania niefunkcjonalne 
\chapter{\customstylechapter{Wymagania niefunkcjonalne}}
% [REQUIREMENT] 10.1. Sprzętowe (w różnych wariantach, w tym dostęp do 
% koniecznych lub alternatywnych nośników danych i peryferiów)
\section{\customstylesection{Sprzętowe wymagania niefunkcjonalne}}
\begin{itemize}
    \item Pamięć 50MB dowolnego typu
    \item Pamięć RAM 4GB (wliczajac system)
    \item Urządzenia peryferyjne: klawiatura, mysz komputerowa
    \item Dowolne urządzenie wyświetlające
\end{itemize}
% [REQUIREMENT] 10.2. Systemowe (systemy operacyjne, zainstalowane środowiska, 
% platformy, pakiety, biblioteki, sterowniki)
\section{\customstylesection{Systemowe wymagania niefunkcjonalne}}
% Wzory: https://www.microsoft.com/pl-pl/windows/Windows-10-specifications
% https://examsoft.com/resources/examplify-minimum-system-requirements/
% https://softwareengineering.stackexchange.com/questions/86863/how-are-minimum-system-requirements-determined
{System Operacyjny Windows 10.}

% [REQUIREMENT] 10.3. Organizacyjne (np. organizacja pracy z systemem, warunki 
% poprawnej pracy przy większej liczbie użytkownikó, stanowisk, obciążeniu sieci,
% konieczność zapewnienia realnego czasu dostępu itd.itp.)
\section{\customstylesection{Organizacyjne wymagania niefunkcjonalne}}
{Projekt aplikacji obejmuje interakcje z pojedynczym użytkownikiem w danym 
momencie, każdy z użytkowników będzie korzystał z własnej instancji bazy 
danych aplikacji która bedzie przechowywana w dowolnej dogodnej określonej 
przez użytkownika pamięci lokalnej określonej w pliku konfiguracyjnym, a także 
zdalnej jeśli wdrożona zostanie funkcjonalność dostępu zdalnego. Aplikacja 
wstępnie nie będzie wymagała stałego dostępu do sieci, jednak w przyszłości 
jej rozwój może zmienić to wymaganie - wymagać wtedy będzie krótkich okresów 
dostępu do sieci. Dostęp do danych będzie wymagany w krótkich okresach zapisu 
danych z pamięci podręcznej aplikacji do bazy oraz odpytania bazy o dane. 
Aplikacja powinna być wykorzystywana na systemach zabezpieczonych przed dostępem
 osób niepowołanych.}

% [REQUIREMENT] 11. Wymagania dotyczące danych (wykaz tabel, relacji, 
% typy i rozmiary pól z uzasadnieniem, inne rodzaje danych w tym logi, hasła)
\chapter{\customstylechapter{Wymagania danych}}
{W tej sekcji spisano wymagania wstępne, jak w każdym projekcie wymagania 
zmieniały się wraz z jego rozwojem w trakcie modelowania i 
wdrażania. Aktualny reprezentatywny stan danych w aplikacji z wyróżnieniem 
wszystkich warstw opisuje rozdział \nameref{Opisy metod}.}

{Użytkownik będzie wprowadzał dane do aplikacji ręcznie lub za pomocą interfejsu
importującego dane w formacie CSV (Comma Separated Values) \cite{CSV}. Dane 
wprowadzone przez użytkownika po walidacji trafią odpowiedniej tabeli w bazie 
danych, natomiast te któe walidacji nie przejdą do zbioru tymczasowego w którym 
użytkownik będzie je mógł poprawić - po przejściu prawidłowo walidacji trafią 
do zbioru docelowego. Wszystkie dane wprowadzone przez użytkownika będą 
traktowane jako ciągi znaków a następnie rzutowane na typy odpowiadające 
docelowym polom w bazie danych do których zostaną przekazane. Wymagania 
danych w bazie oraz aplikacji podzielono na poszczególne sekcje poniżej.}

\section{\customstylesection{Baza danych}}
{Aplikacja wymaga sposobu na trwałe składowanie danych wprowadzonych przez 
użytkownika. W tym celu wraz z aplikacją każdy użytkownik otrzyma dedykowaną 
lokalną instancję bazy danych SQLite \cite{SQLite} do włąsnego użytku. Za 
wyborem tej technologii przemawiają niewielkie wymagania przestrzeni pamięci 
trwałej jakie zajmuje oraz w miarę kompletne wsparcie standardowego języka 
zapytań SQL \cite{SQL}.}

{Baza powinna zawierać tabele do przechowywania danych o przychodach, rachunkach
 i wydatkach oraz wszelkie dane pomocnicze wspomagające normalizację danych. 
Powinna zawierać także widoki wspomagające pracę z danymi obliczające w miarę 
możliwości automatycznie dane analityczne wykorzystywane później w aplikacji. 
Aby zachować spójność danych i raportów dane wprowadzane do bazy powinny być 
odpowiednio walidowane, jednak aby zadbać o ich kompletność użytkownik powinien
mieć możliwość świadomie wprowadzić dane które wymagają poprawy do późniejszej 
obróbki.}

{Szczegóły implementacji wraz z kodem opisane zostaną w sekcji \nameref{Baza danych}}

\section{\customstylesection{Kod aplikacji}}
{Python jest językiem elastycznym, dzięki czemu przyjmuje dane w dowolnej 
formie bez definiowania typu, co pozwala odroczyć projekt klas do momentu 
implementacji a nawet refactoringu programu. W języku Python \cite{Python} 
domyślnie wszystkie zmienne są typu Any co pozwala definiować typy zależnie od 
potrzeb w trakcie tworzenia programu. Wszystkie wczytywane z plików dane 
domyślnie przyjmują typ string lub są traktowane jako pojedyncze bajty 
\cite{Python_read-file}. W efekcie dane wprowadzane przez użytkownika przyjmą 
formę ciągów znaków klasy string, po czym w programie zostaną przekazane do 
konstruktorów klas gdzie zostaną rzutowane na odpowiednie do zadań typy i 
zapisane w polach.}

{Aby ułatwić rozwój aplikacji program powinien zapisywać i przyjmować z pliku 
dane konfiguracyjne w formie list ciągów znaków w formacie "opcja" : "wartość" 
które utworzą obiekt słownika, lub w wariancie bardziej zaawansowanym aplikacja 
może przyjmować konfigurację w postaci pliku JSON \cite{JSON}. Rozwiązanie to 
jest na tyle elastycznie że pozwoli dynamicznie definiować dodatkowe 
konfiguracji w trakcie powstawania projektu - aplikacja zaczyta konfigurację z 
pliku i na jej podstawie utworzy słownik, dzięki czemu dodanie nowego parametru 
sprowadzi się do dopisania linijki tektu do pliku.}

{Aplikacja powinna udostępniać użytkownikowi intuicyjne wizualizacje danych 
dostępne przez interfejs użytkownika. Ponieważ jest to projekt edukacyjny 
aplikacja zostanie zaimplementowana w języku Python \cite{Python} którego autor 
chce się nauczyć. Do implementacji interfejsu użytkownika posłuży biblioteka 
PySimpleGUI \cite{PySimpleGUI} która pozwala tworzyć prosty i responsywny, 
niezależny od platformy interfejs o ograniczonych możliwościach. Jest to adapter
 (tak zwany wrapper) który łączy kilka popularnych bibliotek służących do 
tworzenia interfejsu, udostępnia spójny interfejs dzięki czemu warstwę 
prezentacji można budować wykorzystując tablice złożone z udostępnianych jako 
obiekty widżetów. W fazie prototypu interfejsu nie będzie można konfigurować z 
poziomu aplikacji, natomiast niewykluczone że użytkownik będzie w stanie zmienić
 podstawowe ustawienia edytując dostarczony plik konfiguracyjny. Szczegóły 
implementacji interfejsu opisano w sekcji 
\nameref{Graficzny interfejs użytkownika}.}

{W samym kodzie aplikacji podstawymi strukturami wizualizacji będą kolekcje 
złożone z dat i wartości typu double. Wizualizacje będą przechowywane w 
słownikach, które na podstawie klucza w formacie string służącego także jako 
nazwa wizualizacji zwrócą wartość w formacie string którą będzie zapytanie do 
konkretnej tabeli w bazie danych. Szczegóły implementacji kodu opisano w sekcji 
\nameref{Logika aplikacji}.}


% [REQUIREMENT] 12. Metody pracy, narzędzia i techniki
\chapter{\customstylechapter{Metody pracy, narzędzia i techniki}}
\section{\customstylesection{Metody pracy}}
{Aby dostarczyć minimalny opłacalny produkt \cite{MVP} aplikacja będzie 
rozwijana poprzez wdrażanie wymaganych funkcjonalności w kolejności wynikającej 
z ich priorytetów. Podczas planowania projektu wykorzystana zostanie 
priorytetyzacja MoSCoW \cite{MOSCOW} która polega na określeniu priorytetu za 
pomocą jednej z poniżej wymienionych kategorii, jak miało to miejsce w sekcji 
\nameref{Wymagania funkcjonalne}:
\begin{figure}[H]           %requires float package
    \caption{MoSCoW}
    \label{fig:MoSCoW}
    \centering  
    \includegraphics[width=12cm]{figures/MoSCoW-01.png}
\end{figure}
W fazie prototypu zostaną wdrożone wszystkie funkcjonalności wymagane (M), 
natomiast wdrożenie wszelkich pozostałych kategorii zostanie rozpatrzone w fazie
 rozwoju aplikacji. Plan uwzględnia także cykliczne przeglądy priorytetów aby 
 lepiej dopasować aplikację do potrzeb użytkowników i kierunku rozwoju projektu.
 Zadania rozpisane zostaną w metodologii kanban \cite{Kanban}. Zgodnie z 
zasadami LEAN \cite{LEAN} w każdej iteracji kod będzie dodatkowo refaktoryzowany
 i upraszczany, jeśli zajdzie taka potrzeba i uprości to interfejsy funkcji i 
zwiększy czytelność dane zostaną także zebrane w dedykowane klasy.}

\section{\customstylesection{Narzędzia}}
{Podczas projektowania i wdrożenia aplikacji wykorzystane zostaną narzędzia typu
 Open Source oraz komerycjne dostępne nieodpłatnie dla użytkowników 
indywidualnych.}

{Kategoryzacja MoSCoW \cite{MOSCOW} dla poszczególnych funkcjonalności 
wykonywana będzie na zadaniach zarejstrowanych w tablicy kanban, w serwisie 
serwisu Trello \cite{Trello}. Do stworzenia bazy SQLite \cite{SQLite} posłuży 
aplikacja DataGrid \cite{DataGrid}. Aplikacja Visual Studio Code 
\cite{VSCode} posłuży do pisania kodu w Python \cite{Python} oraz dokumentacji 
w LaTeX \cite{LaTeX}. Do rozwoju dokumentacji oraz kodu aplikacji posłuży 
system kontroli źródła GIT \cite{GIT}, a oba kody źródłowe przechowywane będą w 
osobnych projektach na platformie GitHub \cite{GitHub}.}

\section{\customstylesection{Techniki}}
{W trakcie tworzeznia projektu wykorzystano model przyrostowy 
\cite{Model Przyrostowy} w oparciu o klasyfikację funkcji do wdrożenia metodą 
MoSCoW \cite{MOSCOW}. Ponadto stosowane będą techniki programowania LEAN 
\cite{LEAN} poprawiające czytelność i jakość tworzonego kodu.}

% [REQUIREMENT] 13. Opis głównych klas, metod, obiektów, struktur i algorytmów
% zastosowanych w projekcie (uwzględniając stsosowanie gotowych narzędzi
% obcego autorstwa, w tym open source)
\chapter{\customstylechapter{Opisy metod}} \label{Opisy metod}
\section{\customstylesection{Główne klasy projektu}}
{Projekt składa się z warstw bazy danych oraz graficznego interfejsu aplikacji. 
Baza danych przechowuje dane wprowadzone przez użytkownika w tabelach oraz 
generuje podsumowania i zestawienia w formie widoków. Oba typy obiektów składają
 się na główne klasy projektu. Warstwa graficznego interfejsu użytkownika 
odpowiedzialna jest za interakcję z użytkownikiem oraz interakcję użytkownika z 
bazą - prezentację danych przechowywanych w bazie oraz wizualizacje danych. 
Dodatkowo zbudowano w niej klasy upraszczające interfejs poszczególnych funkcji.}

\section{\customstylesection{Baza danych}}  \label{Baza danych}
{Podczas tworzenia bazy danych przyjęto kilka podstawowych założeń aby utrzymać 
spójną konwencję nazewniczą pól, tabel i widoków. Dzięki niej interfejs bazy 
jest prostszy a pisanie zapytań bardziej intuicyjne co w ogólnym rozrachunku 
powinno ograniczyć nakład pracy wymagany do wdrożenia dodatkowych funkcji.}

\begin{table}[h]
    \footnotesize
    \begin{tabular}{|p{0.2\linewidth}|p{0.73\linewidth}|}  % | draws verical line
    \hline                  % Draw horizontal line
    % & Defines the breaks in the table 
    \customstyletable{Pole} & \customstyletable{Opis} \\
    \hline
    {ID} & {Identyfikator rekordu}\\
    \hline
    {Comment} & {Komentarz użytkownika}\\
    \hline
    {DateTime} & {Znacznik w standardzie daty międzynarodowej ISO 8601 \cite{ISO 8601}}\\
    \hline
    {Amount} & {Koszt, liczba zmiennoprzecinkowa}\\
    \hline
    {Pole specyficzne} & {Główna informacja, różne nazwy w każdej tabeli (Type,Product)}\\
    \hline
    \end{tabular}
    \caption{Konwencja nazewnicza bazy danych }
\end{table}

\begin{figure}[H]           %requires float package
    \caption{Klasy warstwy bazy danych - tabele}
    \label{fig:Klasy warstwy bazy danych - tabele}
    \centering  
    \includegraphics[width=12cm]{figures/Budgeter_Finances-db_Tables_DataGrid.png}
\end{figure}

%-----------------------------TABLES--------------------------------------------
%ADD: Account_data

%[TODO]: Fix - some table span the end of page 
\begin{minipage}{\textwidth}
\begin{lstlisting}[ caption={Tabela ProductTypes},
                    language=SQL,
                    deletekeywords={IDENTITY},
                    deletekeywords={[2]INT},
                    morekeywords={clustered},
                    framesep=8pt,
                    xleftmargin=40pt,
                    framexleftmargin=40pt,
                    frame=tb,
                    framerule=0pt ]
CREATE TABLE [ProductTypes] (
    [ID] 		INTEGER PRIMARY KEY AUTOINCREMENT,
    [Type] 		TEXT 	NOT NULL,
    [Comment] 	TEXT 	DEFAULT NULL
);
\end{lstlisting}
{Tabela ProductTypes zawiera typy produktów zdefiniowane przez użytkownika.}
\end{minipage}

%[TODO] Make function accept 2nd parameter and display caption
%{Tabela ProductTypes zawiera typy produktów zdefiniowane przez użytkownika.}
%\begin{SQLlisting}[{Tabela ProductTypes (v2)}]
%    CREATE TABLE [ProductTypes] (
%        [ID] 		INTEGER PRIMARY KEY AUTOINCREMENT,
%        [Type] 		TEXT 	NOT NULL,
%        [Comment] 	TEXT 	DEFAULT NULL
%    )
%\end{SQLlisting}

\begin{minipage}{\textwidth}
\begin{lstlisting}[ caption={Tabela Products},
    language=SQL,
    deletekeywords={IDENTITY},
    deletekeywords={[2]INT},
    morekeywords={clustered},
    framesep=8pt,
    xleftmargin=40pt,
    framexleftmargin=40pt,
    frame=tb,
    framerule=0pt ]
CREATE TABLE [Products] (
    [ID]        INTEGER PRIMARY KEY AUTOINCREMENT,
    [Product]   TEXT    NOT NULL,
    [TypeID]	INTEGER UNSIGNED, 
    [Comment] 	TEXT    DEFAULT NULL,
FOREIGN KEY([TypeID]) REFERENCES ProductTypes(ID)
);
\end{lstlisting}
{Tabela Products zawiera produkty zdefiniowane przez użytkownika, pole TypeID 
zawiera klucz obcy ID z tabeli ProductTypes.}
\end{minipage}

\begin{minipage}{\textwidth}
\begin{lstlisting}[ caption={Tabela Bills},
    language=SQL,
    deletekeywords={IDENTITY},
    deletekeywords={[2]INT},
    morekeywords={clustered},
    framesep=8pt,
    xleftmargin=40pt,
    framexleftmargin=40pt,
    frame=tb,
    framerule=0pt ]
CREATE TABLE [Bills] (
    [ID]        INTEGER PRIMARY KEY AUTOINCREMENT,
    [Amount]    DOUBLE,
    [Medium]    TEXT,
    [DateTime]  DATETIME,
    [Comment]   TEXT DEFAULT NULL
);
\end{lstlisting}
{Tabela Bills zawiera wydatki stałe wprowadzone przez użytkownika.}
\end{minipage}

\begin{minipage}{\textwidth}
\begin{lstlisting}[ caption={Tabela Income},
    language=SQL,
    deletekeywords={IDENTITY},
    deletekeywords={[2]INT},
    morekeywords={clustered},
    framesep=8pt,
    xleftmargin=40pt,
    framexleftmargin=40pt,
    frame=tb,
    framerule=0pt ]
CREATE TABLE [Income] (
	[ID] 		INTEGER PRIMARY KEY AUTOINCREMENT,
	[Amount]	DOUBLE,
	[Source]	TEXT,
	[Type]		TEXT,
	[DateTime]	DATETIME,
	[Comment]	TEXT DEFAULT NULL
);
\end{lstlisting}
{Tabela Income zawiera przychody wprowadzone przez użytkownika.}
\end{minipage}

\begin{minipage}{\textwidth}
\begin{lstlisting}[ caption={Tabela Expenditures},
    language=SQL,
    deletekeywords={IDENTITY},
    deletekeywords={[2]INT},
    morekeywords={clustered},
    framesep=8pt,
    xleftmargin=40pt,
    framexleftmargin=40pt,
    frame=tb,
    framerule=0pt ]
CREATE TABLE [Expenditures] (
	[ID]	INTEGER,
	[DateTime]	DATETIME,
	[ProductID]	INTEGER UNSIGNED,
	[Amount]	DOUBLE,
	[Comment]	TEXT DEFAULT NULL,
	PRIMARY KEY([ID] AUTOINCREMENT),
	FOREIGN KEY([ProductID]) REFERENCES [Products]([ID])
);
\end{lstlisting}
{Tabela Expenditures zawiera wydatki wprowadzone przez użytkownika.}
\end{minipage}

\begin{minipage}{\textwidth}
\begin{lstlisting}[ caption={Tabela Expenditures\_transitory},
    language=SQL,
    deletekeywords={IDENTITY},
    deletekeywords={[2]INT},
    morekeywords={clustered},
    framesep=8pt,
    xleftmargin=40pt,
    framexleftmargin=40pt,
    frame=tb,
    framerule=0pt ]
CREATE TABLE [Expenditures_transitory] (
	[ID]	INTEGER,
	[DateTime]	DATETIME,
	[ProductID]	INTEGER UNSIGNED,
	[Amount]	DOUBLE,
	[Comment]	TEXT DEFAULT NULL,
	PRIMARY KEY([ID] AUTOINCREMENT),
	FOREIGN KEY([ProductID]) REFERENCES [Products]([ID])
);
\end{lstlisting}
{Tabela Expenditures\_transitory jest tabelą tymczasową, przechowuje wydatki 
wprowadzone przez użytkownika które nie przeszły walidacji. Użytkownik poprawia 
je po czym prawidłowe dane są przenoszone do tabeli Expenditures i  usuwane z 
Expenditures\_transitory.}
\end{minipage}


%------------------------------VIEWS--------------------------------------------
%[TODO]: Add Ledger_comparison, MonthlyCharge

\begin{figure}[H]           %requires float package
    \caption{Klasy warstwy bazy danych - widoki}
    \label{fig:Klasy warstwy bazy danych - widoki}
    \centering  
    \includegraphics[width=12cm]{figures/Budgeter_Finances-db_Views_DataGrid.png}
\end{figure}

\begin{minipage}{\textwidth}
\begin{lstlisting}[ caption={Widok Expenditures\_Enriched},
    language=SQL,
    deletekeywords={IDENTITY},
    deletekeywords={[2]INT},
    morekeywords={clustered},
    framesep=8pt,
    xleftmargin=40pt,
    framexleftmargin=40pt,
    frame=tb,
    framerule=0pt ]
CREATE VIEW [Expenditures_Enriched] AS
SELECT  [EXP].[ID]          as ID,
        [EXP].[DateTime]    as DateTime,
        [EXP].[Amount]      as Amount,
        [PRD].[Product]     as Product,
        [PTY].[Type]        as Type,
        [EXP].[Comment]     as Comment
FROM  [Expenditures]            [EXP]
LEFT JOIN [Products]            [PRD]	
    ON [EXP].[ProductID]=[PRD].[ID]
LEFT JOIN [ProductTypes]        [PTY]
    ON [PRD].[TypeID]=[PTY].[ID]
ORDER BY DateTime;
\end{lstlisting}
{Widok Expenditures\_Enriched prezentuje użytkownikowi czytelne dane z tabeli 
Expenditures wzbogacone o produkty zdefiniowane w tabeli Products i typy z 
tabeli ProductTypes.}
\end{minipage}

\begin{minipage}{\textwidth}
\begin{lstlisting}[ caption={Widok MonthlyExpenditures},
    language=SQL,
    deletekeywords={IDENTITY},
    deletekeywords={[2]INT},
    morekeywords={clustered},
    framesep=8pt,
    xleftmargin=40pt,
    framexleftmargin=40pt,
    frame=tb,
    framerule=0pt ]
CREATE VIEW [MonthlyExpenditures] AS
SELECT 
    SUBSTR([DateTime], 1, 7)    as [Month]
    ,SUM([Amount])              as [Amount]
FROM [Expenditures_Enriched]
GROUP BY [Month]
ORDER BY [Month];
\end{lstlisting}
{Widok MonthlyExpenditures podsumowuje dane o miesięcznych wydatkach użytkownika.}
\end{minipage}

\begin{minipage}{\textwidth}
\begin{lstlisting}[ caption={Widok MonthlyBills},
    language=SQL,
    deletekeywords={IDENTITY},
    deletekeywords={[2]INT},
    morekeywords={clustered},
    framesep=8pt,
    xleftmargin=40pt,
    framexleftmargin=40pt,
    frame=tb,
    framerule=0pt ]
CREATE VIEW [MonthlyBills] AS
SELECT 
    SUBSTR([DateTime], 1, 7)    as [Month]
    ,SUM([Amount])              as [Amount]
FROM [Bills]
GROUP BY [Month]
ORDER BY [Month];
\end{lstlisting}
{Widok MonthlyBills dane o rachunkach bieżących użytkownika w rozrachunku 
miesięcznym na podstawie danych zawartych w tabeli Bills.}
\end{minipage}

\begin{minipage}{\textwidth}
\begin{lstlisting}[ caption={Widok MonthlyIncome},
    language=SQL,
    deletekeywords={IDENTITY},
    deletekeywords={[2]INT},
    morekeywords={clustered},
    framesep=8pt,
    xleftmargin=40pt,
    framexleftmargin=40pt,
    frame=tb,
    framerule=0pt ]
CREATE VIEW [MonthlyIncome] AS
SELECT
	SUBSTR([DateTime], 1, 7)				as Month
	,SUM([Amount])							as Amount
FROM 	[Income]
GROUP BY [Month]
ORDER BY [Month];
\end{lstlisting}
{Widok MonthlyIncome podsumowuje dane o przychodach użytkownika w ujęciu 
miesięcznym.}
\end{minipage}

\begin{minipage}{\textwidth}
\begin{lstlisting}[ caption={Widok MonthlyBilance},
    language=SQL,
    deletekeywords={IDENTITY},
    deletekeywords={[2]INT},
    morekeywords={clustered},
    framesep=8pt,
    xleftmargin=40pt,
    framexleftmargin=40pt,
    frame=tb,
    framerule=0pt ]
CREATE VIEW [MonthlyBilance] AS 
SELECT 
    Strftime('%Y-%m', [DateTime])   as [Month],
    Strftime('%Y',    [DateTime])   as [Year],
    ROUND(SUM([Amount]), 2)         as [Income]
    --Previous_month_income - (bills + expenditures)
FROM (SELECT 
        DATE(Strftime('%Y-%m-01', [DateTime]),[-1 month]) as [DateTime],
        [Amount]  
      FROM [Income]
      UNION SELECT
                [DateTime], 
                -([Amount]) 
            FROM [Expenditures_Enriched]
      UNION SELECT
                [DateTime],
                -([Amount])
            FROM [Bills])
 GROUP BY [Month]
 ORDER BY [Month] DESC;
\end{lstlisting}
{Widok MonthlyBilance podsumowuje bilans miesięczny wydatków i wpływów 
użytkownika w formie pojedynczej liczby.}
\end{minipage}

\begin{minipage}{\textwidth}
\begin{lstlisting}[ caption={Widok Monthly\_common\_products},
    language=SQL,
    deletekeywords={IDENTITY},
    deletekeywords={[2]INT},
    morekeywords={clustered},
    framesep=8pt,
    xleftmargin=40pt,
    framexleftmargin=40pt,
    frame=tb,
    framerule=0pt ]
CREATE VIEW [Monthly_common_products] AS
SELECT * 
FROM (
    SELECT  Strftime('%Y-%m', [DateTime])   AS [Month],
                                               [Product],
            COUNT([Product])                AS [Items], 
            SUM([Amount])                   AS [Sum]
    FROM [Expenditures_Enriched]
    GROUP BY [Product], [Month]
    ORDER BY [Month] DESC, [Sum] DESC
) WHERE [Items]>=4;
\end{lstlisting}
{Widok Monthly\_common\_products podsumowuje dane o produkach które użykownik 
kupował najczęściej każdego miesiąca. Uwzględnia wyłącznie produkty które 
zakupiono 4 razy - liczbę wybrano arbitralnie metodą kolejnych przybliżeń aby 
otrzymać zadowalający wynik.}
\end{minipage}

\begin{minipage}{\textwidth}
\begin{lstlisting}[ caption={Widok Monthly\_Expenditures\_by\_Type},
    language=SQL,
    deletekeywords={IDENTITY},
    deletekeywords={[2]INT},
    morekeywords={clustered},
    framesep=8pt,
    xleftmargin=40pt,
    framexleftmargin=40pt,
    frame=tb,
    framerule=0pt ]
CREATE VIEW [Monthly_Expenditures_by_Type] AS 
SELECT	Strftime('%Y-%m', [DateTime]) as [Month],
        Strftime('%Y',    [DateTime]) as [Year],
        ROUND(SUM([Amount]), 2)       as [Sum],
        [Type]
FROM (SELECT 
        [DateTime], 
        [Type], 
        [Amount] 
      FROM [Expenditures_Enriched])
GROUP BY [Type], [Month]
ORDER BY [Month] DESC, [Sum] DESC;
\end{lstlisting}
{Widok Monthly\_Expenditures\_by\_Type podsumowuj dane o typach produktów 
zakupionych przez użytkownika w ujęciu miesięcznym.}
\end{minipage}

\begin{minipage}{\textwidth}
\begin{lstlisting}[ caption={Widok Temp\_check},
    language=SQL,
    deletekeywords={IDENTITY},
    deletekeywords={[2]INT},
    morekeywords={clustered},
    framesep=8pt,
    xleftmargin=40pt,
    framexleftmargin=40pt,
    frame=tb,
    framerule=0pt ]
CREATE VIEW [Temp_check] 
    (Temp_ID, Temp_Product, Product_ID)
AS
SELECT *
FROM (SELECT 
        Expenditures_transitory.ID      as [Temp_ID],
        Expenditures_transitory.Product as [Temp_Product],
        Products.ID                     as [Product_ID]
      FROM [Expenditures_transitory]
      LEFT OUTER JOIN [Products]
      ON [Expenditures_transitory].[Product]==[Products].[Product]
)
WHERE [Product_ID] IS NULL;
\end{lstlisting}
{Widok Temp\_check weryfikuje poprawność danych wprowadzonych przez użytkownika.}
\end{minipage}

\begin{minipage}{\textwidth}
\begin{lstlisting}[ caption={Widok Products\_to\_fix},
    language=SQL,
    deletekeywords={IDENTITY},
    deletekeywords={[2]INT},
    morekeywords={clustered},
    framesep=8pt,
    xleftmargin=40pt,
    framexleftmargin=40pt,
    frame=tb,
    framerule=0pt ]
CREATE VIEW [Products_to_fix] AS
SELECT *
FROM [Expenditures_Enriched]
WHERE [Product] IN (NULL,
                    'UNKNOWN')
    OR [Comment] LIKE '%[TODO]%';
\end{lstlisting}
{Widok Products\_to\_fix zawiera dane wprowadzone przez użytkownika poprawnie i 
oznaczone jako dane do uzupełnienia specjalnymi etykietami - wartością w polu 
PRODUCT=UNKNOWN lub [TODO] w komentarzu.}
\end{minipage}

\begin{minipage}{\textwidth}
    \begin{lstlisting}[ caption={Widok TypeSummary},
        language=SQL,
        deletekeywords={IDENTITY},
        deletekeywords={[2]INT},
        morekeywords={clustered},
        framesep=8pt,
        xleftmargin=40pt,
        framexleftmargin=40pt,
        frame=tb,
        framerule=0pt ]
    CREATE VIEW IF NOT EXISTS [TypeSummary] AS
    SELECT
        [ProductTypes].[ID]                         AS [ID]
        ,[ProductTypes].[Type]                    AS [Type]
        ,[ProductTypes].[Comment]              AS [Comment]
        ,IFNULL([Summary].[Amount], 0)          AS [Amount]
     ,IFNULL([Summary].[Bought Times], 0) AS [Bought Times]
        ,[Summary].[FirstBought]           AS [FirstBought]
        ,[Summary].[LastBought]             AS [LastBought]
        ,IFNULL([Summary].[Common], 'Absent')   AS [Common]
    FROM [ProductTypes]
    LEFT JOIN (SELECT 
        *
        ,(CASE WHEN ([Bought Times]>(
            SELECT 
                AVG([Bought Times]) AS [Average] 
            FROM (SELECT
                    [Type]
                    ,Round(SUM([Amount]), 2) AS [Amount]
                    ,COUNT([DateTime]) AS [Bought Times]
                    ,MAX([DateTime])     AS [LastBought]
                    ,MIN([DateTime])    AS [FirstBought]
                FROM [Expenditures_Enriched]
                GROUP BY [Type]
                ORDER BY [Bought Times] DESC))) 
        then 'Common' else 'Uncommon' end) as [Common]
        
    FROM (	SELECT
               [Type]
               ,Round(SUM([Amount]), 2) AS [Amount]
               ,COUNT([DateTime])       AS [Bought Times]
               ,MAX(DateTime)           AS [LastBought]
               ,MIN(DateTime)           AS [FirstBought]
            FROM [Expenditures_Enriched]
            GROUP BY [Type]
            ORDER BY [Bought Times] DESC))
    AS [Summary]
    ON [ProductTypes].[Type]=[Summary].[Type];
    \end{lstlisting}
    {Widok TypeSummary podsumowuj dane o typach produktów użytkownika.}
    \end{minipage}
    
    \begin{minipage}{\textwidth}
        \begin{lstlisting}[ caption={Widok ProductSummary},
            language=SQL,
            deletekeywords={IDENTITY},
            deletekeywords={[2]INT},
            morekeywords={clustered},
            framesep=8pt,
            xleftmargin=40pt,
            framexleftmargin=40pt,
            frame=tb,
            framerule=0pt ]
    CREATE VIEW IF NOT EXISTS [ProductSummary] AS
    SELECT
        [Products].[ID]                             AS [ID]
        ,[Products].[Product]                  AS [Product]
        ,[Products].[TypeID]                    AS [TypeID]
        ,[Products].[Comment]                  AS [Comment]
        ,IFNULL([Summary].[Amount], 0)          AS [Amount]
     ,IFNULL([Summary].[Bought Times], 0) AS [Bought Times]
        ,[Summary].[FirstBought]           AS [FirstBought]
        ,[Summary].[LastBought]             AS [LastBought]
        ,IFNULL([Summary].[Common], 'Absent')   AS [Common]
    FROM [Products]
    LEFT JOIN (SELECT 
        *
        ,(CASE WHEN ([Bought Times]>(
            SELECT 
                AVG([Bought Times]) AS [Average] 
            FROM (SELECT
                    [Product]
                    ,Round(SUM([Amount]), 2) AS [Amount]
                    ,COUNT([DateTime]) AS [Bought Times]
                    ,MAX([DateTime])     AS [LastBought]
                    ,MIN([DateTime])    AS [FirstBought]
                FROM [Expenditures_Enriched]
                GROUP BY [Product]
                ORDER BY [Bought Times] DESC))) 
        then 'Common' else 'Uncommon' end) as [Common]
        
    FROM (	SELECT
               [Product]
               ,Round(SUM([Amount]), 2) AS [Amount]
               ,COUNT([DateTime])       AS [Bought Times]
               ,MAX(DateTime)           AS [LastBought]
               ,MIN(DateTime)           AS [FirstBought]
            FROM [Expenditures_Enriched]
            GROUP BY [Product]
            ORDER BY [Bought Times] DESC))
    as [Summary]
    ON [Products].[Product]=[Summary].[Product];
    \end{lstlisting}
    {Widok ProductSummary podsumowujący dla użytkownika statystyki produktów.}
    \end{minipage}

%------------------------- Application layer ----------------------------------- 
\section{\customstylesection{Logika aplikacji}} \label{Logika aplikacji}
{W toku prac stopniowo wykorzystywane w projekcie zmienne w formie kolekcji 
słowników  zamieniano w klasy które spajają dane. Wyłoniły się one w wyniku 
refaktoryzacji i tworzenia abstrakcji upraszczających interfejs funkcji. Klasy 
projektowano tak, by były w miarę możliwości oczywiste i zrozumiałe, co ma na 
celu poprawić czytelność i zrozumiałość kodu.}

\begin{minipage}{\textwidth}
    \begin{lstlisting}[ caption={Klasa Database},
        language=Python,
        deletekeywords={IDENTITY},
        deletekeywords={[2]INT},
        morekeywords={clustered},
        framesep=8pt,
        xleftmargin=40pt,
        framexleftmargin=40pt,
        frame=tb,
        framerule=0pt ]
class Database():
    def __init__(self, 
                 fullpath,
                 schema,
                 selects, 
                 inserts,
                 updates):
        self.fullpath = fullpath
        self.schema = schema
        self.selects = selects
        self.inserts = inserts
        self.updates = updates
    \end{lstlisting}
{Obiekty klasy Database zawierają komplet informacji wymaganych do interakcji z
 bazą danych wykorzystwaną w aplikacji. Pole fullpath to w pełni kwalifikowana
 ścieżka do bazy danych, schema jest kolekcją obiektów typu string która 
przechowuje schemat bazy danych. Pozostałe pola: selects, inserts, updates to 
słowniki które pozwalają po nazwie odwołać się do odpowiednio zapytań (SELECT), 
dodawania rekordów do tabel (INSERT), oraz aktualizacji danych w tabelach 
(updates).}
\end{minipage}

\begin{minipage}{\textwidth}
    \begin{lstlisting}[ caption={Klasa ChartSelect},
        language=Python,
        deletekeywords={IDENTITY},
        deletekeywords={[2]INT},
        morekeywords={clustered},
        framesep=8pt,
        xleftmargin=40pt,
        framexleftmargin=40pt,
        frame=tb,
        framerule=0pt ]
class ChartSelect():
    def __init__(self,
                 database,
                 select,
                 label
                ):
        self.database=str(database),
        self.select=str(select),
        self.label=str(label)
    \end{lstlisting}
{Obiekty klasy ChartSelect posiadają trzy atrybuty typu string: database
 przechowuje w pełni kwalifikowaną ścieżkę do bazy danych aplikacji, select 
to zapytanie SQL do bazy, natomiast pole label to etykieta wykresu danych 
wyświetlanego użytkownikowi.}
\end{minipage}

\begin{minipage}{\textwidth}
    \begin{lstlisting}[ caption={Klasa Chart},
        language=Python,
        deletekeywords={IDENTITY},
        deletekeywords={[2]INT},
        morekeywords={clustered},
        framesep=8pt,
        xleftmargin=40pt,
        framexleftmargin=40pt,
        frame=tb,
        framerule=0pt ]
class Chart():
    def __init__(self,
                 selects,
                 caption):
        self.selects = selects
        self.caption = caption
    \end{lstlisting}
{Obiekty klasy Chart definiują dane do wizualizacji. Atrybut caption przyjmuje 
wartości typu string wyświetlane jako nagłówek wizualizacji, natomiast atrybut 
selects jest listą obiektów typu ChartSelect - zbioru zapytań które zostaną 
wyświetlone w ramach pojedynczej wizualizacji.}
\end{minipage}

\begin{minipage}{\textwidth}
    \begin{lstlisting}[ caption={Klasa CellEdition},
        language=Python,
        deletekeywords={IDENTITY},
        deletekeywords={[2]INT},
        morekeywords={clustered},
        framesep=8pt,
        xleftmargin=40pt,
        framexleftmargin=40pt,
        frame=tb,
        framerule=0pt ]
class CellEdition():
    def __init__(self,
                 table, 
                 ID, 
                 field, 
                 newvalue,
                 oldvalue): 
        self.table = table
        self.ID = ID
        self.field = field
        self.newvalue = newvalue
        self.oldvalue = oldvalue

    def __repr__(self): 
        return "Table % s modified. ID: % s field: % s oldvalue: % s newvalue: % s" % (self.table, 
                 self.ID, 
                 self.field, 
                 self.newvalue,
                 self.oldvalue)
    \end{lstlisting}
{Obiekty klasy CellEdition przechowują dane edytowanego przez użytkownika 
używającego interfejsu aplikacji rekordu bazy danych. Każdy obiekt przechowuje w
 polach odpowiednio:}
\begin{itemize}
    \item table - tabelę któej dotyczy zmiana
    \item ID - identyfikator modyfikowanego rekordu
    \item field - pole które jest zmieniane
    \item newvalue - nowa wartość pola
    \item oldvalue - wartość pola przed zmianą
\end{itemize}
{Funkcja składowa \_\_repr\_\_ formatuje dane które zawiera obiekt do postaci 
tekstu. Obecnie nieużywana, możliwe że w późniejszych etapach zostanie 
wykorzystana do rejestrowania zdarzenia w logu aplikacji.}
\end{minipage}

\section{\customstylesection{Graficzny interfejs użytkownika}} 
\label{Graficzny interfejs użytkownika}
{Graficzny interfejs użytkownika (GUI, Graphical User interface) aplikacji 
utworzono z wykorzystaniem biblioteki PySimpleGUI \cite{PySimpleGUI}. Dzięki 
temu interfejs definiowany jest w postaci kolekcji jak listy, lub listy list, 
obiektów klas zawartych w bibliotece - jako przykład przedstawiono opcje listy 
rozwijanej na poniższym listingu. %TODO: Figure out how to reference listings \ref{Dropdown}. 
Aby zapewnić responsywność interfejs budowany jest w kilku etapach, a cała 
budowa wydzielona do specjalnych funkcji generujących wywoływanych później 
zależnie od potrzeb. Funkcje opisane są w dalszej części w sekcji 
\nameref{Metody projektu}.}

\begin{minipage}{\textwidth}
    \begin{lstlisting}[ caption={Lista rozwijana przykład definicji interfejsu},
        language=Python,
        deletekeywords={IDENTITY},
        deletekeywords={[2]INT},
        morekeywords={clustered},
        framesep=8pt,
        xleftmargin=40pt,
        framexleftmargin=40pt,
        frame=tb,
        framerule=0pt ]
    menu = [['Visualizations', 
                ['Most common products', 
                 'Income summary',
                 'Monthly Bilance',
                 'TopTypeMonthly',
                 'Type'
                    ,[types],
                 'Product'
                    ,[products]]],
            ['Browse data',
                #TODO: Add views as uneditable
                ['Expenditures',
                 'Bills',
                 'Income',
                 'Types' ,
                 'Products',]],
            ['Options',
                [#'Configure',
                 'Change Theme',
                    [themes],
                'Version',
                'About...',
                'Manual']] #TODO: Wishful thinking - built in manual
            ]
    \end{lstlisting}
\end{minipage}

{Interfejs przedstawiono poglądowo na poniższyczh grafikach.}
%{Jego iteracyjny rozwój zaprezentowano na przykładzie zakładki Visualizations.} %TODO: optional
\begin{figure}[H]           %requires float package
    \caption{Wizualizacja danych}
    \label{fig:Wizualizacja danych}
    \centering  
    \includegraphics[width=12cm]{figures/Interface_Visualizations_v0.3.png}
\end{figure}

\begin{figure}[H]           %requires float package
    \caption{Opcje}
    \label{fig:Opcje}
    \centering  
    \includegraphics[width=12cm]{figures/Interface_Options_v0.3.png}
\end{figure}

\begin{figure}[H]           %requires float package
    \caption{Import danych}
    \label{fig:Import danych}
    \centering  
    \includegraphics[width=12cm]{figures/Interface_Browse_Import_v0.3.png}
\end{figure}

\begin{figure}[H]           %requires float package
    \caption{Dodawanie rekordu}
    \label{fig:Dodawanie rekordu}
    \centering  
    \includegraphics[width=12cm]{figures/Interface_Browse_AddRecord_v0.3.png}
\end{figure}

\section{\customstylesection{Metody projektu}} 
\label{Metody projektu}
{lorem ipsum}

\section{\customstylesection{Obiekty projektu}}
{lorem ipsum}

\section{\customstylesection{Struktury projektu}}
{lorem ipsum}

\section{\customstylesection{Algorytmy projektu}}
{lorem ipsum}

% [REQUIREMENT] 22. Bibliografia - wykaz wszystkich źródeł
\begin{thebibliography} {books}
\bibitem{wiki_ekonomia} Wikipedia, Nauki Ekonomiczne \raggedright\url{
    https://pl.wikipedia.org/wiki/Nauki_ekonomiczne}
\bibitem{gus_sytuacja_budzetowa} Główny Urząd Statystyczny \raggedright\url{
    https://stat.gov.pl/obszary-tematyczne/warunki-zycia/dochody-wydatki-i-warunki-zycia-ludnosci/sytuacja-gospodarstw-domowych-w-2021-r-w-swietle-badania-budzetow-gospodarstw-domowych,3,21.html}
\bibitem{o24_budzetowanie} Opcje24, Budzetowanie \raggedright\url{
    https://www.opcje24h.pl/budzetowanie-przewodnik-planowanie-budzetu/}
\bibitem{MOSCOW} Product Plan, MOSCOW Prioritetization \raggedright\url{
    https://www.productplan.com/glossary/moscow-prioritization/}
\bibitem{MatrycaEisenhowera}Praca.pl, Matryca Eisenhowera - czym jest, zasada, prioryteryzacja zadań \raggedright\url{
    https://www.praca.pl/poradniki/rynek-pracy/matryca-eisenhowera-czym-jest,zasada,prioryteryzacja-zadan_pr-2012.html}
\bibitem{MVP} Wikipedia, Minimal Viable Product \raggedright\url{
    https://en.wikipedia.org/wiki/Minimum_viable_product}
\bibitem{ISO 8601} NASA.gov, A summary of the international standard date and time notation \raggedright\url{
    https://fits.gsfc.nasa.gov/iso-time.html}
\bibitem{CSV} Y. Shafranovich, SolidMatrix Technologies, Inc., Common Format and MIME Type for Comma-Separated Values (CSV) Files \raggedright\url{
    https://www.rfc-editor.org/rfc/rfc4180}
\bibitem{SQLite} sqlite.org, SQLite \raggedright\url{
    https://www.sqlite.org/index.html}
\bibitem{SQL} wikipedia.org, SQL - Structured Query Language \raggedright\url{
    https://en.wikipedia.org/wiki/SQL}
\bibitem{Python} python.org, Python \raggedright\url{
    https://www.python.org/}
\bibitem{PySimpleGUI} pysimplegui.org, PySimpleGUI Python GUIs for Humans \raggedright\url{
    https://www.pysimplegui.org/en/latest/}
\bibitem{JSON} json.org, Introducing JSON \raggedright\url{
    https://www.json.org/json-en.html}
\bibitem{Python_read-file} pythonspot.com, Python tutorials, How to Read a File in Python \raggedright\url{
    https://pythonspot.com/read-file/}
\bibitem{Trello} Atlassian, Trello.com \raggedright\url{
    https://trello.com/}
\bibitem{StarUML} MKLabs Co.,Ltd, StarUML \raggedright\url{
    https://staruml.io/}
\bibitem{LaTeX} The LaTeX Project \raggedright\url{
    https://www.latex-project.org/}
\bibitem{VSCode} Microsoft, Visual Studio Code \raggedright\url{
    https://code.visualstudio.com/}
\bibitem{DataGrid} JetBrains, DataGrid \raggedright\url{
    https://www.jetbrains.com/datagrip/}
\bibitem{Kanban} Lean Action PLan, Kanban – układ nerwowy sterowania produkcją w koncepcji Lean Manufacturing \raggedright\url{
    https://leanactionplan.pl/kanban/}
\bibitem{LEAN} Wikipedia, Lean software development \raggedright\url{
    https://pl.wikipedia.org/wiki/Lean_software_development}
\bibitem{GIT} git-scm.com, git \raggedright\url{
    https://git-scm.com/}
\bibitem{GitHub} https://github.com/ \raggedright\url{
    https://github.com/}
\bibitem{Model Przyrostowy} Wikipedia, Model Przyrostowy \raggedright\url{
    https://pl.wikipedia.org/wiki/Model_przyrostowy}

\end{thebibliography}




%https://github.com/MarcinNowak94/budgeter
%https://github.com/MarcinNowak94/DatabaseShenanigans

% [REQUIREMENT] 21. Spisy ilustracji (spis obiektów graficznych, diagramów, tabel...)
\listoffigures
\listoftables
\lstlistoflistings

\end{document}


% ------------------------------ Docummentation -------------------------------

% ------ Project deadline 2023-01-29 ------------------------------------------
% ------ Part 1: Due 2022-10-21 

% ------ Part 2: Due 2022-11-18 
% [REQUIREMENT] 15. Przebieg uruchamiania projektu (być może na różnych 
% platformach, konfiguracjach...)
% [REQUIREMENT] 16. Przebieg testowania projektu (rodzaje i metody przeprowadzonych
% testów)
% [REQUIREMENT] 17. Wnioski z przebiegu testowania (wykryte defekty, wrażliwość
% na specyficzne dane, błędy ukryte i niewidoczne dla użytkownika, sytuacje
% niejednoznaczne itp.)
% [REQUIREMENT] 18. Konserwacja systemu
% [REQUIREMENT] 19. Podsumowanie i alternatywne sposoby stworzenia projektu
% (po zdobytym doświadczeniu, przy dostępie do innych narzędzi, przy innej wizji...)
% [REQUIREMENT] 20. Dokumentacja dla użytkownika (Podręczni kużytkownika)
% [REQUIREMENT] 20.1. Przeznaczenie i główne możliwości systemu
% [REQUIREMENT] 20.2. Podstawowe wymagania
% [REQUIREMENT] 20.3. Opis instalacji/uruchamiania
% [REQUIREMENT] 20.4. Kompletny opis działających funkcji (menu, opis interface...),
% formatów danych, obsługi błędów użytkowania, zakresów danych 
% [REQUIREMENT] 20.5. Podręcznik administratora/użytkownika systemu/gościa
% [REQUIREMENT] 20.6. Spostrzeżenia i zalecenia do użytkowania projektu
% [REQUIREMENT] 20.7. Wykryte błędy w działaniu


% DONE
% [REQUIREMENT] 1. Strona tytułowa
% [REQUIREMENT] 2. Spis treści
% [REQUIREMENT] 3. Wprowadzenie do tematyki projektu
% [REQUIREMENT] 4. Zamierzony cel projektu
% [REQUIREMENT] 5. Wstępne założenia i uwarunkowania, w których 
% projekt będzie powstawał  -----------------------
% [REQUIREMENT] 6. Założone ograniczenia (ramy czasowe, umiejętności) 
% i możliwość ewaluacji projektu
% [REQUIREMENT] 7. Chronologiczny plan pracy (ujecie przyjetego modelu 
% projektowania i faz projektowania)
% [REQUIREMENT] 8. POWYŻSZĄ CZĘŚĆ DOKUMENTACJI ODDAJEMY PRZED REALIZACJĄ PROJEKTU
% [REQUIREMENT] 9. Wymagania funkcjonalne (szczegółowey wykaz wszystkich funkcji 
% oprogramowania) jakie funkcjonalności oprogramowania chcemy dostarczyć, można 
% nie zdążyć z dostarczeniem części, lub dopisać dodatkowe dodane w trakcie
% [REQUIREMENT] 10. Wymagania niefunkcjonalne
% [REQUIREMENT] 10.1. Sprzętowe (w różnych wariantach, w tym dostęp do 
% koniecznych lub alternatywnych nośników danych i peryferiów)
% [REQUIREMENT] 10.2. Systemowe (systemy operacyjne, zainstalowane środowiska, 
% platformy, pakiety, biblioteki, sterowniki)
% [REQUIREMENT] 10.3. Organizacyjne (np. organizacja pracy z systemem, warunki 
% poprawnej pracy przy większej liczbie użytkownikó, stanowisk, obciążeniu sieci,
% konieczność zapewnienia realnego czasu dostępu itd.itp.)
% [REQUIREMENT] 11. Wymagania dotyczące danych (wykaz tabel, relacji, 
% typy i rozmiary pól z uzasadnieniem, inne rodzaje danych w tym logi, hasła)
% [REQUIREMENT] 12. Metody pracy, narzędzia i techniki
% [REQUIREMENT] 13. Opis głównych klas, metod, obiektów, struktur i algorytmów
% zastosowanych w projekcie (uwzględniając stsosowanie gotowych narzędzi
% obcego autorstwa, w tym open source)
% [REQUIREMENT] 14. POWYŻSZĄ CZĘŚĆ DOKUMENTACJI ODDAJEMY W TRAKCIE TWORZENIA
% [REQUIREMENT] 23. Przypisy dolne, stopki (nr stron), nagłówki...
% [REQUIREMENT] 22. Bibliografia - wykaz wszystkich źródeł
% [REQUIREMENT] 21. Spisy ilustracji (spis obiektów graficznych, diagramów, tabel...)

% --------------------------------- Purgatory ----------------------------------
% Orphaned snippets of texts placed here untill you decide what to do with them
% -----
% domowym i udostępnienie podstawowych informacji finansowych Niestety jedak 
% zakres wiedzy w zakresie finansów którą posiada przeciętny obywatel jest dość 
% niski, w połączeniu z obecnie panującą trudną sytuacją gospodarczą w wielu 
% miejscach na świecie część ludzi nie radzi sobie z dopinaniem budżetu 
% domowego. Jest to jednak dziedzina dość skomplikowana, a tym samym trudna dla 
% przeciętnego obywatela.