% Docummentation: 
% - https://www.latex-project.org/help/documentation/
% - https://docs.w3cub.com/latex/

\documentclass[a4paper,10pt]{report}

\usepackage{polski}         % Polish diacretic signs
\usepackage[utf8]{inputenc} % required for international characters
\usepackage{hyperref}       % urls and hyperlinks
\usepackage{xurl}           % break urls
\usepackage{microtype}      % improve justification
\usepackage{enumitem}       % compact lists
\usepackage{graphicx}       % Include graphics
\usepackage{wrapfig}        % wrap text aroung graphics
\usepackage{fancyhdr}       % Customize page layout
\usepackage{index}          % Create an index
\usepackage{setspace}       % Spacing
\usepackage{float}          % Forcing figure placement
\usepackage{tabularray}     % Tables with wrapping
\makeindex
\graphicspath{ {./figures/} }

% ------------------------------ Custom Commands ------------------------------
% Usage: \command\{text}  
\newcommand{\customstyletitle}[1]{\Huge{\textbf{#1}}}
\newcommand{\customstylechapter}[1]{\large{\textit{#1}}}
\newcommand{\customstylesection}[1]{\textbf{\textit{#1}}}
\newcommand{\customstylesidenote}[1]{\Small{\textbf{#1}}}
\newcommand{\customstyletable}[1]{\footnotesize{\textbf{#1}}}


\newcommand{\HRule}{\rule{\linewidth}{0.5mm}} % horizontal lines

% --------------------------- documment starts here ---------------------------

% environment
\begin{document}

% Define pagestyle
% [REQUIREMENT] 23. Przypisy dolne, stopki (nr stron), nagłówki...
\pagestyle{fancy}
\fancyhf{}          %clears default headers and footers
\renewcommand{\headrulewidth}{2pt}
\renewcommand{\footrulewidth}{1pt}

\fancypagestyle{mychapterpage}{%
    %\fancyhead[LE]{\leftmark}
    %\fancyhead[RO]{\rightmark}
    \fancyfoot[RO,LE]{\thepage}
    \renewcommand{\headrulewidth}{2pt}
    \renewcommand{\footrulewidth}{1pt}
}


% https://texblog.org/2013/09/16/multiple-page-styles-with-fancyhdr/
%Redefine chapter by adding fancy as the chapter title page page-style
\makeatletter
    \let\stdchapter\chapter
    \renewcommand*\chapter{%
    \@ifstar{\starchapter}{\@dblarg\nostarchapter}}
    \newcommand*\starchapter[1]{%
        \stdchapter*{#1}
        \thispagestyle{mychapterpage}
        \fancyfoot[RO,LE]{\thepage}
        \markboth{\MakeUppercase{#1}}{}
    }
    \def\nostarchapter[#1]#2{%
        \stdchapter[{#1}]{#2}
        \thispagestyle{mychapterpage}
        \fancyfoot[RO,LE]{\thepage}
    }
\makeatother


% [REQUIREMENT] 1. Strona tytułowa
\begin{titlepage}
	%---Headings------------------------------------------	
	\begin{center}
    \begin{onehalfspace}
    \textsc{\LARGE{WYŻSZA SZKOŁA TECHNOLOGII INFORMATYCZNYCH W KATOWICACH}}\\
    \end{onehalfspace}
    \textsc{\large{WYDZIAŁ INFORMATYKI}}\\
	\textsc{\large{KIERUNEK: INFORMATYKA}}\\
    \end{center}
    
    %---Author--------------------------------------------
	\begin{flushleft}
    \textsc{Nowak Marcin}\\[0cm]
    \textsc{Nr Albumu 08255}\\[0cm]
    \textsc{Studia niestacjonarne}\\[0cm]
    \end{flushleft}
	
    %---Title---------------------------------------------
	\begin{center}
    \HRule\\[0.4cm]
	{\customstyletitle{Projekt aplikacji wspomagającej zarządzanie budżetem}}\\[0.4cm] 
    \HRule\\[1.5cm]
    \end{center}
	
    %---Description----------------------------------------
	\begin{flushright}
        \textsc{Przedmiot: Projekt Systemu Informatycznego}\\[0cm]
        \textsc{pod kierunkiem}\\[0cm]
        \textsc{mgr. Jacek Żywczok}\\[0cm]
        \textsc{W roku akademickim 2022/23}\\[0cm]
    \end{flushright}
 
	%---Date & logo---------------------------------------
	\vfill                  % Position the date lower
	\begin{center}
    {Katowice 2022}\\	    % \today
	\includegraphics[width=0.2\textwidth]{WSTI-logo.jpg}\\[1cm]
	\end{center}
\end{titlepage}

% [REQUIREMENT] 2. Spis treści
\renewcommand*\contentsname{Spis treści}
\tableofcontents                    % prints automatical table of contents

% ---------------------------------- Content ----------------------------------

% [REQUIREMENT] 3. Wprowadzenie do tematyki projektu
\chapter{\customstylechapter{Wprowadzenie do tematyki projektu}}
{Finanse są dziedziną nauki ekonomicznej zajmującą się rozporzadzaniem pieniędzmi
 \cite{wiki_ekonomia}. Nauka ta w podobnym zakresie a różnej skali 
wykorzystywana jest tak przez rządy, przedsiębiorstwa jak i zwykłych obywateli - 
w efekcie jest to dziedzina o stosunkowo prostych podstawach jednak niesamowicie
skomplikowana w każdym zakresie w którym chętna osoba zadecyduje się ją zagłębić. 
Wiedza z zakresu finansów staje się szczególnie przydatna podczas gdy na rynku 
panuje trudna sytuacja ekonomiczna, w takich warunkach nierzadko decyduje ona o 
jakości oraz stanie życia poszczególnych osób fizycznych jak i całych 
przedsiębiorstw a nawet krajów.}
%
% [REQUIREMENT] 4. Zamierzony cel projektu
\chapter{\customstylechapter{Zamierzony cel projektu}}
{Celem projektu jest ułatwienie zarządzania finansami i budżetem poprzez 
uproszczenie analizy wpływów i wydatków dzięki wizualizacji trendów, 
automatycznej kategoryzacji wydatków i wpływów oraz predefiniowanym 
zestawieniom. Docelowymi odbiorcami aplikacji są użytkownicy domowi oraz 
średnie lub małe przedsiębiorstwa. Użytkownik po wprowadzeniu danych będzie 
w stanie w łatwy sposób zobrazować sytuację finansową osobistą lub 
przedsiębiorstwa co pozwoli bardziej świadomie podejmować dalsze decyzje 
finansowe, planować budżet, łatwo identyfikować obszary które wymagają 
usprawnień czy ogólną obserwację trendów.}
%
% [REQUIREMENT] 5. Wstępne założenia i uwarunkowania, w których 
% projekt będzie powstawał  -----------------------
\chapter{\customstylechapter{Wstępne założenia i uwarunkowania}}
\section{\customstylesection{Założenia}}
{Początkowo użytkownik będzie wprowadzał dane do aplikacji samodzielnie poprzez 
dedykowany interfejs. Aplikacja zadba o jakość danych przyjmując jednak 
oznaczając i pomijając dane błędne, niepełne lub niepewne które zaprezentuje w 
dedykowanej zakładce gdzie użytkownik będzie mieć możliwość ich poprawy. 
Użytkownik będzie w stanie wybrać zestaw predefiniowanych typów i kategorii 
obiektów lub utworzyć i edytować własne. Aplikacja będzie udostępniać 
predefiniowane wizualizacje, wliczając możliwość wizualizacji określonego 
przez użytkownika obiektu.}
%
\section{\customstylesection{Uwarunkowania}}
{Celem projektu jest dostarczenie minimalnego opłacalnego produktu \cite{MVP}, 
obecnie pozostałe funkcjonalności zostaną pominięte z różnych przyczyn jak 
ograniczony czas wdrożenia, zakres umiejętności technicznych autora czy fakt że 
jest to projekt w głównej mierze edukacyjny. Termin wdrożenia wyklucza bardziej 
zaawansowane funcjonalności, jako że jest to projekt edukacyjny znajomość 
technologii będzie budowana w trakcie jego rozwoju co wpłynie między innymi na 
ograniczenia systemowe. Aplikacja będzie także z zasady obsługiwać wyłącznie 
pojedynczego użytkownika, a zawarte w niej dane będa  przechowywane wyłącznie 
lokalnie. Pominięte zostanie także automatyczne pobieranie danych z interfejsów 
innych aplikacji lub w formie ekstrakcji danych ze skanowanych dokumentów czy 
kodów EAN lub QR towarów. Aplikacja nie będzie także udostępniać żadnego rodzaju
 interfejsu programistycznego (API). W momencie zakończenia projektu wszystkie 
dane użytkownika przechowywane będą w pojedynczym miejscu, w przyszłości może 
jednak zajść potrzeba rozdzielenia danych w aplikacji od konfiguracji 
użytkownika. Interfejs aplikacji będzie statyczny bez możliwości zmiany przez 
użytkownika.}
%
% [REQUIREMENT] 6. Założone ograniczenia (ramy czasowe, umiejętności) 
% i możliwość ewaluacji projektu
\chapter{\customstylechapter{Założone ograniczenia i możliwosci ewaluacji projektu}}
{W aplikacji utworzony zostanie panel administracyjny prezentujący użytkownikowi 
dane statystyczne prezentujące ilość, zakres i jakość danych a także sugerujące 
kolejny krok ich usprawnienia. Standard danych w aplikacji dopasowany zostanie 
do wiodącego globalnego standardu danych w obrębie tej samej tematyki. Typy 
obiektów będzie można grupować na kilku poziomach aby ułatwić użytkownikowi 
zarządzanie danymi i uprościć wizualizacje. Dla zaawansowanych użytkowników 
może okazać się przydatna możliwość definiowania i zapisywania własnych 
wizualizacji i raportów statystycznych - wymagać to będzie jednak implementacji 
dedykowanego modułu. Kolejnym obecnie pominiętym aspektem jest zabudowanie reguł
 przeprowadzających dogłębną analizę statystyczną danych które otwierają dalsze 
możliwości rozwoju oprogramowania.}

{Funkcjonalności importu i eksportu danych ze standardowych formatów będzie 
przydatna dla użytkownika podczas korzystania z projektu, wymaga określenia 
odpowiedniego formatu i standardu plików co może zająć sporo czasu dlatego 
zostały uznane za dodatkowe i nie zostaną wdrożone w początkowej fazie projektu.}
% [REQUIREMENT] 7. Chronologiczny plan pracy (ujecie przyjetego modelu 
% projektowania i faz projektowania)
\chapter{\customstylechapter{Plan pracy}}
{Prace nad projektem prowadzone będą w formie listy zadań do zrealizowania 
którym przypisane zostaną priorytety metodą MoSCoW \cite{MOSCOW} lub Matrycy 
Eisenhowera. Przewidywany plan pracy nad projektem prezentuje się następująco:
\setitemize{noitemsep,topsep=0pt,parsep=0pt,partopsep=0pt}
\begin{enumerate}
    \item Spis założeń w dokumentacji wstępnej
    \begin{itemize}
        \item Założenia wstępne
        \item Spis wymagań każdego typu
        \item Przegląd rynku pod kątem dostępnych rozwiązań
        \item Określenie metodologii pracy
        \item Dokumentacja modelowania
        \item Dokumentacja uruchomieniowa projektu
        \item Przeprowadzone testy
        \item Instrukcja obsługi dla użytkownika
        \item Retrospekcja
    \end{itemize}
    \item Modelowanie 
    \begin{itemize}
        \item Utworzenie słownika modelowanej domeny
        \item Określenie wymaganych kontenerów
        \item Określenie wymaganych encji i atrybutów
        \item Określenie wymaganych ograniczeń danych
        \item Modelowanie powiazań encji
    \end{itemize}
    \item Wybór technologii
    \begin{itemize}
        \item Wspierane systemy i wersje
        \item Wybór języka
        \item Biblioteki interfejsu użytkownika
        \item Sposób przechowywania danych
        \item Instalator, aktualizacja i utrzymanie 
    \end{itemize}
    \item Wstępne wdrożenie
    \begin{itemize}
        \item Utworzenie struktur bazy danych
        \item Wypełnienie danymi testowymi
        \item Podstawowe triggery i widoki
        \item Projekt interfejsu użytkownika
        \item Szkielet interfejsu użytkownika
        \item Połączenie interfejsu z bazą danych
        \item Podstawowa wizualizacja
        \item Iteracyjne uzupełnienie interfejsu i bazy o dodatkowe funkcje
        \item Usprawnienia
    \end{itemize}
    \item Testy rozwiązania
    \begin{itemize}
        \item Utworzenie danych testowych
        \item Określenie spodziewanych wyników
        \item Porównanie wyników oczekiwanych z otrzymanymi 
    \end{itemize}
    \item Iteracyjne usprawnienia projektu i uzupełnianie dokumentacji
    \item Retrospekcja
    \begin{itemize}
        \item Przydatność gotowej aplikacji
        \item Wady i zalety podejścia
        \item Sprawność rozwiązań
        \item Sprawność technologii
        \item Spis wniosków
    \end{itemize}
\end{enumerate}
}

% [REQUIREMENT] 9. Wymagania funkcjonalne (szczegółowey wykaz wszystkich funkcji 
% oprogramowania) jakie funkcjonalności oprogramowania chcemy dostarczyć, można 
% nie zdążyć z dostarczeniem części, lub dopisać dodatkowe dodane w trakcie
\chapter{\customstylechapter{Wymagania funkcjonalne}}
{Zestawienie funkcji które powinien spełniać program, wraz z informacją któe 
z nich zostały spełnione. Nagłówki z powodu objętości zostały skrócone, legenda:}

{PRIO - Priorytet w jednej z kategorii MOSCOW \cite{MOSCOW}}

{IMPL - Oznaczenie czy wdrożono funkcjonalność}

% Wrapping as per: https://stackoverflow.com/questions/790932/how-to-wrap-text-in-latex-tables
\begin{table}[h]
    \footnotesize
    \begin{tabular}{|p{0.2\linewidth}|p{1cm}|l|p{0.6\linewidth}|}  % | draws verical line
    % \usepackage{booktabs} provides different line thicknesses
    % \toprule, \midrule, \bottomrule
    \hline                  % Draw horizontal line
        
    % & Defines the breaks in the table 
    \customstyletable{Funkcjonalność} & \customstyletable{PRIO} & \customstyletable{IMPL}& \customstyletable{Opis} \\
    \hline
    {Plik konfiguracji} & {M} & {-} & {Osobny plik konfiguracyjny}\\
    \hline
    {Panel konfiguracyjny} & {S} & {-} & {Osobny panel konfiguracyjny}\\
    \hline
    {Dostęp zdalny} & {C} & {-} & {Dostęp do zdalnych baz danych}\\
    \hline
    {Definiowanie produktów} & {M} & {-} & {Definiowanie produktów}\\
    \hline
    {Definiowanie przychodów} & {M} & {-} & {Definiowanie przychodów}\\
    \hline
    {Definiowanie typów produktów} & {M} & {-} & {Definiowanie typów produktów}\\
    \hline
    {Definiowanie typów przychodów} & {M} & {-} & {Definiowanie typów przychodów}\\
    \hline
    {Podsumowanie wydatków} & {M} & {-} & {Okresowe podsumowanie wydatków}\\
    \hline
    {Podsumowanie przychodów} & {M} & {-} & {Okresowe podsumowanie przychodów}\\
    \hline
    {Statystyki typów} & {C} & {-} & {Statystyki wydatków na dany typ produktu}\\
    \hline
    {Statystyki produktów} & {C} & {-} & {Statystyki wydatków na dany produkt}\\
    \hline
    {Bilans okresowy} & {M} & {-} & {Okresowy bilnas zysków i strat}\\
    \hline
    {Trendy} & {W} & {-} & {Predykcja trendów wydatkó i wpływów}\\
    \hline
    {Porady} & {W} & {-} & {Porady dla użytkownika dotyczące usprawnień budżetu}\\
    \hline
    {Instalator} & {C} & {-} & {Prosty instalator aplikacji}\\
    \hline
    {Aktualizacje} & {W} & {-} & {Automatyczne sprawdzanie wersji i aktualizacja}\\
    \hline
    {Instalator} & {C} & {-} & {Prosty instalator aplikacji}\\
    \hline
    {Import danych} & {C} & {-} & {Import danych w standardowym formacie}\\
    \hline
    {Walidacja danych} & {C} & {-} & {Potwierdzenie jakości danych}\\
    \hline
    {Eksport danych} & {C} & {-} & {Import danych do standardowego formatu}\\
    \hline
    {Wiele użytkowników} & {W} & {-} & {Wsparcie dla wielu użytkowników jednocześnie}\\
    \hline
    {Personalizacja interfejsu} & {W} & {-} & {Personalizacja interfejsu użytkownika}\\
    \hline
    \end{tabular}
    \caption{Wymagania funkcjonalne}
\end{table}

% [REQUIREMENT] 10. Wymagania niefunkcjonalne 
\chapter{\customstylechapter{Wymagania niefunkcjonalne}}
% [REQUIREMENT] 10.1. Sprzętowe (w różnych wariantach, w tym dostęp do 
% koniecznych lub alternatywnych nośników danych i peryferiów)
\section{\customstylesection{Sprzętowe wymagania niefunkcjonalne}}
{Pamięć 50MB dowolnego typu, pamięć RAM 2GB, klawiatura, mysz komputerowa, 
dowolny monitor, opcjonalne połączenie z siecią internet.},
% [REQUIREMENT] 10.2. Systemowe (systemy operacyjne, zainstalowane środowiska, 
% platformy, pakiety, biblioteki, sterowniki)
\section{\customstylesection{Systemowe wymagania niefunkcjonalne}}
% Wzory: https://www.microsoft.com/pl-pl/windows/Windows-10-specifications
% https://examsoft.com/resources/examplify-minimum-system-requirements/
% https://softwareengineering.stackexchange.com/questions/86863/how-are-minimum-system-requirements-determined
{System Operacyjny Windows 10, Python3.0, MySQL.}

% [REQUIREMENT] 11.3. Organizacyjne (np. organizacja pracy z systemem, warunki 
% poprawnej pracy przy większej liczbie użytkownikó, stanowisk, obciążeniu sieci,
% konieczność zapewnienia realnego czasu dostępu itd.itp.)
\section{\customstylesection{Organizacyjne wymagania niefunkcjonalne}}
{Aplikacja wspierać bedzie diałanie z wyłącznie jednym użytkownikiem 
jednocześnie, każdy z użytkowników będzie korzystał z własnej instancji bazy 
danych w aplikacji która bedzie przechowywana w dowolnej dogodnej określonej 
przez użytkownika pamięci lokalnej, także zdalnej jeśli wdrożona zostanie 
funkcjonalność dostępu zdalnego. Aplikacja wstępnie nie będzie wymagała stałego 
dostępu do sieci, jednak w przyszłości rozwój jej funkcjonalności może zmienić 
to wymaganie, wymagać wtedy będzie krótkich okresów dostępu do sieci. Dostęp do 
danych będzie wymagany w krótkich okresach zapisu danych z pamięci podręcznej 
aplikacji do bazy oraz odpytania bazy o dane. Aplikacja powinna być 
wykorzystywana na systemach zabezpieczonych przed dostępem osób niepowołanych.}

% [REQUIREMENT] 11. Wymagania dotyczące danych (wykaz tabel, relacji, 
% typy i rozmiary pól z uzasadnieniem, inne rodzaje danych w tym logi, hasła)
\chapter{\customstylechapter{Wymagania danych}}
{Sekcja będzie uzupełniana wraz z rozwojem projektu, w trakcie modelowania i 
wdrażania tak, aby odzwierciedlała faktyczny stan aplikacji.}

{Użytkownik będzie wprowadzał dane do aplikacji ręcznie lub za pomocą interfejsu
importującego dane w formacie CSV (Comma Separated Values). Dane wprowadzone 
przez użytkownika trafią do zbioru tymczasowego z któreg po walidacji 
potwierdzone prawidłowe dane trafią do zbioru docelowego. Dane które nie przejdą
 walidacji pomyślnie pozostaną w zbiorze tymczasowym gdzie użytkownik będzie 
mógł je poprawić, uzupełnić lub usunąć.}

% [REQUIREMENT] 12. Metody pracy, narzędzia i techniki
\chapter{\customstylechapter{Metody pracy, narzędzia i techniki}}
\section{\customstylesection{Metody pracy}}
{Aby dostarczyć minimalny opłacalny produkt \cite{MVP} aplikacja będzie 
rozwijana poprzez wdrażanie wymaganych funkcjonalności w kolejności wynikającej 
z ich priorytetów. W projekcie będzie wykorzystywana priorytetyzacja MoSCoW 
\cite{MOSCOW} która polega na określeniu priorytetu za pomocą jednej z kategorii:
\begin{figure}[H]           %requires float package
    \caption{MoSCoW}
    \label{fig:MoSCoW}
    \centering  
    \includegraphics[width=12cm]{MoSCoW-01.png}
\end{figure}
W fazie projektu zostaną wdrożone wszystkie funkcjonalności wymagane, natomiast  
wszelkie pozostałe kategorie zostaną wdrożone w miarę możliwości. Plan 
uwzględnia także cykliczne przeglądy priorytetów aby lepiej dopasować aplikację 
do potrzeb użytkowników i kierunku rozwoju projektu. Zadania rozpisane zostaną 
w metodologii kanban}

\section{\customstylesection{Narzędzia}}
{Podczas projektowania i wdrożenia aplikacji wykorzystane zostaną narzędzia typu
 Open Source oraz komerycjne dostępne nieodpłatnie dla użytkowników 
indywidualnych.}

{Kategoryzacja MoSCoW dla poszczególnych funkcjonalności wykonywana będzie na 
zadaniach zarejstrowanych w tablicy kanban, do metodologii kanban wykorzystany 
zostanie serwis Trello. Model encji w aplikacji zostanie wykonany w aplikacji 
StarUML. Do stworzenia bazy SQLite posłuży aplikacja DB Browser for SQLite. 
Aplikacja Visual Studio Code posłuży do pisania kodu w Python oraz dokumentacji 
w LaTeX.}

\section{\customstylesection{Techniki}}
{lorem ipsum}

% [REQUIREMENT] 13. Opis głównych klas, metod, obiektów, struktur i algorytmów
% zastosowanych w projekcie (uwzględniając stsosowanie gotowych narzędzi
% obcego autorstwa, w tym open source)
\chapter{\customstylechapter{Opisy metod}}
\section{\customstylesection{Główne klasy projektu}}
{lorem ipsum}

\section{\customstylesection{Metody projektu}}
{lorem ipsum}

\section{\customstylesection{Obiekty projektu}}
{lorem ipsum}

\section{\customstylesection{Struktury projektu}}
{lorem ipsum}

\section{\customstylesection{Algorytmy projektu}}
{lorem ipsum}

% [REQUIREMENT] 22. Bibliografia - wykaz wszystkich źródeł
\begin{thebibliography} {books}
\bibitem{wiki_ekonomia} Wikipedia, Nauki Ekonomiczne \raggedright\url{
    https://pl.wikipedia.org/wiki/Nauki_ekonomiczne}
\bibitem{gus_sytuacja_budzetowa} Główny Urząd Statystyczny \raggedright\url{
    https://stat.gov.pl/obszary-tematyczne/warunki-zycia/dochody-wydatki-i-warunki-zycia-ludnosci/sytuacja-gospodarstw-domowych-w-2021-r-w-swietle-badania-budzetow-gospodarstw-domowych,3,21.html}
\bibitem{o24_budzetowanie} Opcje24, Budzetowanie \raggedright\url{
    https://www.opcje24h.pl/budzetowanie-przewodnik-planowanie-budzetu/}
\bibitem{MOSCOW} Product Plan, MOSCOW Prioritetization \raggedright\url{
    https://www.productplan.com/glossary/moscow-prioritization/}
\bibitem{MVP} Wikipedia, Minimal Viable Product \raggedright\url{
    https://en.wikipedia.org/wiki/Minimum_viable_product}
\end{thebibliography}

\listoffigures

\end{document}



% ------------------------------ Docummentation -------------------------------

% ------ Project deadline 2023-01-29 ------------------------------------------
% ------ Part 1: Due 2022-10-21 

% ------ Part 2: Due 2022-11-18 
% [REQUIREMENT] 14. POWYŻSZĄ CZĘŚĆ DOKUMENTACJI ODDAJEMY W TRAKCIE TWORZENIA

% [REQUIREMENT] 15. Przebieg uruchamiania projektu (być może na różnych 
% platformach, konfiguracjach...)
% [REQUIREMENT] 16. Przebieg testowania projektu (rodzaje i metody przeprowadzonych
% testów)
% [REQUIREMENT] 17. Wnioski z przebiegu testowania (wykryte defekty, wrażliwość
% na specyficzne dane, błędy ukryte i niewidoczne dla użytkownika, sytuacje
% niejednoznaczne itp.)
% [REQUIREMENT] 18. Konserwacja systemu
% [REQUIREMENT] 19. Podsumowanie i alternatywne sposoby stworzenia projektu
% (po zdobytym doświadczeniu, przy dostępie do innych narzędzi, przy innej wizji...)
% [REQUIREMENT] 20. Dokumentacja dla użytkownika (Podręczni kużytkownika)
% [REQUIREMENT] 20.1. Przeznaczenie i główne możliwości systemu
% [REQUIREMENT] 20.2. Podstawowe wymagania
% [REQUIREMENT] 20.3. Opis instalacji/uruchamiania
% [REQUIREMENT] 20.4. Kompletny opis działających funkcji (menu, opis interface...),
% formatów danych, obsługi błędów użytkowania, zakresów danych 
% [REQUIREMENT] 20.5. Podręcznik administratora/użytkownika systemu/gościa
% [REQUIREMENT] 20.6. Spostrzeżenia i zalecenia do użytkowania projektu
% [REQUIREMENT] 20.7. Wykryte błędy w działaniu
% [REQUIREMENT] 21. Spisy ilustracji (spis obiektów graficznych, diagramów, tabel...)


% DONE
% [REQUIREMENT] 1. Strona tytułowa
% [REQUIREMENT] 2. Spis treści
% [REQUIREMENT] 3. Wprowadzenie do tematyki projektu
% [REQUIREMENT] 4. Zamierzony cel projektu
% [REQUIREMENT] 5. Wstępne założenia i uwarunkowania, w których 
% projekt będzie powstawał  -----------------------
% [REQUIREMENT] 6. Założone ograniczenia (ramy czasowe, umiejętności) 
% i możliwość ewaluacji projektu
% [REQUIREMENT] 7. Chronologiczny plan pracy (ujecie przyjetego modelu 
% projektowania i faz projektowania)
% [REQUIREMENT] 8. POWYŻSZĄ CZĘŚĆ DOKUMENTACJI ODDAJEMY PRZED REALIZACJĄ PROJEKTU
% [REQUIREMENT] 23. Przypisy dolne, stopki (nr stron), nagłówki...


% --------------------------------- Purgatory ----------------------------------
% Orphaned snippets of texts placed here untill you decide what to do with them
% -----
% domowym i udostępnienie podstawowych informacji finansowych Niestety jedak 
% zakres wiedzy w zakresie finansów którą posiada przeciętny obywatel jest dość 
% niski, w połączeniu z obecnie panującą trudną sytuacją gospodarczą w wielu 
% miejscach na świecie część ludzi nie radzi sobie z dopinaniem budżetu 
% domowego. Jest to jednak dziedzina dość skomplikowana, a tym samym trudna dla 
% przeciętnego obywatela.